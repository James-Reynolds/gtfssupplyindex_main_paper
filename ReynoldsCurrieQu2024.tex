\documentclass[preprint, 3p,
authoryear]{elsarticle} %review=doublespace preprint=single 5p=2 column
%%% Begin My package additions %%%%%%%%%%%%%%%%%%%

\usepackage[hyphens]{url}

  \journal{Transport Geography?} % Sets Journal name

\usepackage{graphicx}
%%%%%%%%%%%%%%%% end my additions to header

\usepackage[T1]{fontenc}
\usepackage{lmodern}
\usepackage{amssymb,amsmath}
% TODO: Currently lineno needs to be loaded after amsmath because of conflict
% https://github.com/latex-lineno/lineno/issues/5
\usepackage{lineno} % add
\usepackage{ifxetex,ifluatex}
\usepackage{fixltx2e} % provides \textsubscript
% use upquote if available, for straight quotes in verbatim environments
\IfFileExists{upquote.sty}{\usepackage{upquote}}{}
\ifnum 0\ifxetex 1\fi\ifluatex 1\fi=0 % if pdftex
  \usepackage[utf8]{inputenc}
\else % if luatex or xelatex
  \usepackage{fontspec}
  \ifxetex
    \usepackage{xltxtra,xunicode}
  \fi
  \defaultfontfeatures{Mapping=tex-text,Scale=MatchLowercase}
  \newcommand{\euro}{€}
\fi
% use microtype if available
\IfFileExists{microtype.sty}{\usepackage{microtype}}{}
\usepackage[]{natbib}
\bibliographystyle{plainnat}

\usepackage{graphicx}
\ifxetex
  \usepackage[setpagesize=false, % page size defined by xetex
              unicode=false, % unicode breaks when used with xetex
              xetex]{hyperref}
\else
  \usepackage[unicode=true]{hyperref}
\fi
\hypersetup{breaklinks=true,
            bookmarks=true,
            pdfauthor={},
            pdftitle={Social needs for transport and gaps in transit service: new GTFS tools},
            colorlinks=false,
            urlcolor=blue,
            linkcolor=magenta,
            pdfborder={0 0 0}}

\setcounter{secnumdepth}{5}
% Pandoc toggle for numbering sections (defaults to be off)


% tightlist command for lists without linebreak
\providecommand{\tightlist}{%
  \setlength{\itemsep}{0pt}\setlength{\parskip}{0pt}}




\usepackage{subfig}
\usepackage{booktabs}
\usepackage{longtable}
\usepackage{array}
\usepackage{multirow}
\usepackage{wrapfig}
\usepackage{float}
\usepackage{colortbl}
\usepackage{pdflscape}
\usepackage{tabu}
\usepackage{threeparttable}
\usepackage{threeparttablex}
\usepackage[normalem]{ulem}
\usepackage{makecell}
\usepackage{xcolor}



\begin{document}


\begin{frontmatter}

  \title{Social needs for transport and gaps in transit service: new
GTFS tools}
    \author[Public Transport Research Group (PTRG)]{James Reynolds%
  %
  \fnref{1}}
   \ead{james.reynolds@monash.edu} 
    \author[Public Transport Research Group (PTRG)]{Graham Currie%
  \corref{cor1}%
  \fnref{2}}
   \ead{graham.currie@monash.edu} 
    \author[Public Transport Research Group (PTRG)]{Yanda Qu%
  %
  \fnref{3}}
   \ead{yanda.qu@monash.edu} 
      \affiliation[Public Transport Research Group (PTRG)]{
    organization={Public Transport Research Group (PTRG), Institute of
Transport Studies, Department of Civil Engineering Engineering, Monash
University},addressline={Clayton
Campus},city={Melbourne},postcode={3800},state={Victoria},country={Australia},}
    \cortext[cor1]{Corresponding author}
    \fntext[1]{Research Fellow}
    \fntext[2]{Professor}
    \fntext[3]{PhD Student}
  
  \begin{abstract}
  This is the abstract.

  It consists of two paragraphs.
  \end{abstract}
    \begin{keyword}
    keyword1 \sep 
    keyword2
  \end{keyword}
  
 \end{frontmatter}

\section{Introduction}\label{introduction}

A common role for transit is providing some mobility for those who
cannot otherwise drive themselves, so they can access activities and
services beyond walking distance \citep{Currie:2016aa}. Age, disability,
socio-economic status, lack of a vehicle or many other factors might
make someone reliant on transit for some or all of their travel. As
well, lower proportions of younger people are obtaining a driving
license than was the case previously \citep{delbosc2013causes},
suggesting an increased need for transit coverage in the future.

Vertical social-equity perspectives on transport policy-making relate to
supporting those who are disadvantaged \citep{Litman:2016aa} and might
suggest providing at least some transit, and probably more than just a
minimum, where there are higher social needs for transport. Along these
lines \citet{Currie2003Hobart}, \citet{Currie2004Gap};
\citet{Currie2007Identifying} and \citet{currie2010identifying}
developed an approach for identifying spatial gaps in transit supply
related to social needs for transport, and applied it to a case study of
Melbourne in 2006.

However, there does not appear to have been much further use or
development of this spatial analysis technique. As well, it is unclear
whether the spatial patterns identified for Melbourne in 2006 have
changed in the intervening years or are representatvie of other places.
This may in part be because, until recently, schedules were not
available in a consistent electronic format, meaning that assessing
transit supply was a large task requiring bespoke sourcing, cleaning and
analysis of data for each operator. Nowadays, however, more than 10,000
transit agencies publicly release timetable data in the General Transit
Feed Specification (GTFS) format \citep{GTFS}. Such standardisation
allows Google Maps and other online platforms to provide transit-related
outputs for any place publishing a feed, but tools for using GTFS data
to examine spatial patterns and gaps in transit supply with respect to
social needs for transport do not appear to be readily available. This
gap, and the lack of direct follow up to \citet{Currie2003Hobart},
\citet{Currie2004Gap}, \citet{Currie2007Identifying} and
\citet{currie2010identifying}, provide motivation for this research.

The objectives of this research are: (1) to develop tools for
undertaking needs-gap analysis using GTFS datasets; and (2) to better
understand spatial patterns of gaps between social needs for transport
and transit supply, including whether those reported in
\citet{Currie2007Identifying} and \citet{currie2010identifying} are
representative of the current situation in Melbourne and elsewhere. This
paper reports the development of a new R package (gtfssupplyindex) with
tools for using the \citet{Currie2007Identifying} and
\citet{currie2010identifying} analysis approach with GTFS datasets. Also
presented in this paper are results for Melbourne in 2016 and 2021,
matching the most recent censuses, for comparison to the 2006 results
reported in \citet{currie2010identifying}\footnote{The wider programme
  of research includes examination of spatial gaps in other cities, so
  as to better understand whether patterns in Melbourne are
  representative of other places. However, this paper is limited to
  examining Melbourne in 2016 and 2021 only. Results for other cities
  will be reported elsewhere.}.

The remainder of this paper is structured as follows: the next section
outlines the research context; Section 3 describes the study
methodology; results are presented in Section 4 and discussed in Section
5; and limitations of this study and directions for future research are
discussed in Section 6, which concludes the paper.

\section{Research context}\label{research-context}

There are many metrics available for assessing transit services.
Examples include: those in the \emph{Transit Cooperative Research
Program (TCRP) Report 88: a guidebook for developing
performance-measurement systems} \citep{Ryus:2003aa}; and those used
across benchmarking databases and programs such as by
\citet{Florida-Transit-Information-System:2018aa}, \citet{UITP:2015aa}
and \citet{Imperial-College-London:2023aa}. The Fielding Triangle
\citep{FieldingGordonJ1987Mpts} provides a framework for combining
indicators of service inputs, outputs and consumption to describe cost
efficiency, cost effectiveness and service effectiveness. More broadly:
\citet{Litman:2003ab} and \citet{Litman:2016aa} discuss the traffic,
mobility, accessibility, social equity, strategic planning and other
rational decision-making perspectives underlying many transport
indicators; \citet{Reynolds:2017ah} extends these into models of how
institutionalism, incrementalism and other public policy analysis
concepts might apply to transit prioritisation;
\citet{GuzmanLuisA.2017Aeit} developed a measure of accessibility in the
context of policy development and social equity for Latin American Bus
Rapid Transit (BRT) networks; and
\citet{Creutzig2020streetspaceallocation} introduced street space
allocation metrics based around ten ethical principles.

Many such metrics, however, may be difficult to calculate, understand or
use, especially for those who are not planners, engineers or other
technical specialists. Where pre-calculated transit metrics are
immediately available, it may not be possible to independently generate
scores or assess proposed system changes. Contrasting examples are
provided by:

\begin{itemize}
\item
  \emph{Transit Score} \citep{WalkScore:2023tg}, which are readily
  available online. The meaning of the metric appears simple, with the
  highest possible score of 100 representing the sort of transit
  accessibility experienced in the center of New York. However, the
  algorithm is secret, and scores cannot be calculated independently.
\item
  The \emph{Transit Capacity and Quality of Service Manual (TCQSM)}
  \citep{TCQSM:2013}, which provides metrics for many different aspects
  of a transit system. Scores are consistently grouped into five
  categories, providing a simple ranking from very good to not very
  good, and can be independently calculated (with sufficient data).
  However, there are many metrics, which may complexity to using TCQSM
  scores for communication with the public, politicians or other
  stakeholders.
\end{itemize}

GTFS datasets have allowed the application of such metrics to many
transit systems. The \emph{Transit Score} website provides an example,
while \citet{Wong:2013aa} provides another in reporting the distribution
of various \emph{TCQSM} metrics across 50 USA transit operators. Code
used in the \citet{Wong:2013aa} analysis is available for those who
might wish to produce similar analysis for other locations or time
periods. The lack of a similar code base for calculating spatial gaps in
transit supply with respect to social needs provides motivation for the
research reported in this paper.

\subsection{The Transit Suppy Index}\label{the-transit-suppy-index}

A generalized form of the SI equation, adapted from
\citet{currie2010identifying}, is:

\[SI_{area, time} = \sum{\frac{Area_{Bn}}{Area_{area}}SL_{n, time}}\]

where:

\begin{itemize}
\item
  \(SI_{area, time}\) is the Supply Index for the area of interest and a
  given period of time;
\item
  \(Area_{Bn}\) is the buffer area for each stop (n) within the area of
  interest (in \citet{currie2010identifying} this was based on a radius
  of 400 metres for bus and tram stops, and 800 metres for railway
  stations);
\item
  \(Area_{area}\) is the area of the area of interest; and
\item
  \(SL_{n,time}\) is the number of transit arrivals for each stop for a
  given time period.
\end{itemize}

\begin{figure}

{\centering \subfloat[Distribution of Transit Supply\label{fig:Currie_map_SI-1}]{\includegraphics[width=0.5\linewidth]{graphics/Currie2010SI} }\subfloat[Distibution of Social Need for Transport\label{fig:Currie_map_SI-2}]{\includegraphics[width=0.5\linewidth]{graphics/Currie2010Needs} }\newline\subfloat[Supply Index and Composite Needs Index scores\label{fig:Currie_map_SI-3}]{\includegraphics[width=0.5\linewidth]{graphics/Currie2010chart} }\subfloat[CCDs with Very High needs and Very Low or Zero supply\label{fig:Currie_map_SI-4}]{\includegraphics[width=0.5\linewidth]{graphics/Currie2010gap} }\newline

}

\caption{Melbourne 2006 Social Needs-Gap Results. Source: Currie (2010)}\label{fig:Currie_map_SI}
\end{figure}

\citet{currie2010identifying} reported SI scores for Census Collection
Districts (CCDs) across Melbourne in 2006. These were used to categorise
service levels into seven groups, as shown in Figure
\ref{fig:Currie_map_SI}(a). General patterns were identified, being:
more transit supply in the inner and middle suburbs, and along passenger
railway lines; and outer areas tending to have very low supply or no
transit at all.

\subsection{Social need and needs-gap}\label{social-need-and-needs-gap}

\citet{currie2010identifying} assessed the social need for transport
across Melbourne using a composite index including: the Australian
Bureau of Statistics' (ABS') Index of Related Socio-Economic
Advantage/Disadvantage (IRSAD) and a transport needs index derived from
eight weighted indicators. The spatial distribution of this composite
social needs index, reproduced in Figure \ref{fig:Currie_map_SI}(b),
showed that areas of above average, high and very high social needs were
located in: some outer suburbs, particularly in the east and south-east;
and in some middle suburbs in the south-east, north and west.

\citet{currie2010identifying} compared social needs and Supply Index
scores (Figure \ref{fig:Currie_map_SI}(c)) and identified areas with
very high transport needs, but very low or zero transit supply (Figure
\ref{fig:Currie_map_SI}(d)), where service gaps might be of particular
concern. Most of these were located in the outer suburbs of Melbourne in
the north-east, south-east and south, although there were also some
pockets in the middle suburbs in the west, north and south east.
Overall,\\
``8.2\% of Melbourne residents ha(d) `very high' needs but `zero', `low'
or `very low' public transport supply''\citep{currie2010identifying}.

This approach does not appear to have been adopted widely in practice or
by researchers. Our suspicion is that while the SI has a relatively
simple formula and requires only geographic and timetable data to
calculate, a lack of software tools to complete the analysis may be
partly why it has not been more widely adopted. The case-based methods
used in this research to develop such tools are discussed in the
following.

\section{Methodology}\label{methodology}

Case approaches can be particularly useful when research questions are
about `how' or `why', but researchers lack control of events (preventing
experiments)\citep{Yin2009aa}. Here research questions relate to: (1)
how to use GTFS data to assess gaps between social needs and transit
supply, and (2) how and why spatial patterns might have changed since
those reported for 2006 in \citet{currie2010identifying}. There is no
ability here to control events so a case research approach appears well
suited to this study.

When using a case research approach there is also a need to address the
``duality criterion'', being the need to seek generalisable findings
while at the same time being grounded in the context of only a small
number of cases (allowing the study to go into greater
depth)\citep{Denscombe2007aa, Ketokivi2014aa}. Here the approach taken
has been to develop a package of tools for calculating the SI from GTFS
data using the R programming language \citep{R-base}, so that the
duality criterion is addressed by developing generic software functions
that might be applied to other GTFS feeds, beyond the Melbourne case
reported in this paper. The recommendations of \citet{wickham2023r}
informed the package setup and development approach. Various existing
packages and code examples were relied upon including: the sf package
\citep{R-sf} for geospatial analysis; the tidyverse
\citep{tidyverse2019}; gtfstools \citep{R-gtfstools}; and tidytransit
\citep{R-tidytransit}.

There are, however, limitations to the generalisability of this research
with respect to changes since 2006 and the patterns reported in
\citet{currie2010identifying}(research question 2).
\citet{Currie2007Identifying} and \citet{currie2010identifying} reported
results for Melbourne only, and while \citet{Currie2003Hobart} and
\citet{Currie2004Gap} reported results for Hobart and Adelaide these
used earlier versions of the needs-gap assessment methodology, for which
software tools have not been developed here. While this study does seek
to findings about changes in spatial patterns of social needs and
transit supply that are generalisable to more than just Melbourne, a
lack of 2006 results for other cities may limit the extent to which
these findings might be confidently considered representative of changes
in other places\footnote{Other parts of this research programme will
  examine changes over the last 10-15 years in other cities (where GTFS
  data is available). The focus of this paper, however, is on Melbourne.}.

\subsection{Code developement}\label{code-developement}

Code was developed and tested on the Mornington Peninsula Tourist
Railway GTFS feed. This was selected primarily for convenience, given
that the authors are familiar with the surrounding geography and that
the feed covers only a small number of trips across just three stations
(thereby facilitating hand verification of outputs). ABS data was used
to define areas of interest, and was sourced via the absmapsdata package
\citep{R-absmapsdata}.

\subsection{Melbourne Case Study}\label{melbourne-case-study}

The methodological literature provides guidance on case selection and
discusses various theoretical sampling approaches, including sampling
for critical, particularly revelatory and/or representative cases
\citep{Eisenhardt1989aa, Yin2009aa, Denscombe2007aa, Eisenhardt2007TBfC}.
\citet{Yin2009aa} notes selection of a case to allow longitudinal study,
and this is the primary reason for selecting Melbourne here, so as to
facilitate comparison with the 2006 results reported in
\citet{currie2010identifying}. As such, SI scores were calculated using
the same Census Collection Districts (CCDs) used by
\citet{currie2010identifying}, but for the weeks starting the day of the
2016 and 2021 censuses. The Victorian GTFS feed, published by Public
Transport Victoria (PTV), was used, with historical feeds sourced from
\citet{transitfeeds_victoria:2023aa}.

Unfortunately, it is not possible to obtain 2016 or 2021 social
disadvantage data for CCDs, as the ABS no longer releases data using
this geographic scheme. Instead, population and other statistics are now
released for Statistical Area (SA) zones, under a hierarchical structure
ranging from SA1s (\textasciitilde400 people) to SA4s (parts of a city
or region)\citep{ABSmaps}. As such, SI scores have also been calculated
for SA1s, to facilitate use of ABS data to identify needs-gaps.

The same Transport Supply categorizations have been used as in
\citet{currie2010identifying}\footnote{Zero, Very Low, Low, Below
  average, Above average, High and Very High.}. As well, this study
adopts a similar approach to measuring social disadvantage as used in
\citet{currie2010identifying}, using: the ABS' IRSAD and a transport
needs index\footnote{The same need indicators and weightings used in
  \citet{currie2010identifying} were adopted, although \$799 or lower
  per week was used as the threshold for low income households rather
  than \$499 to account for inflation (as per the Reserve Bank of
  Australia's online inflation calculator).}. A composite needs
indicator was derived based on the IRSAD and the transport needs index,
again as per the \citet{currie2010identifying} approach. However,
changes to ABS reporting means that the composite needs indicator used
by \citet{currie2010identifying} cannot be exactly replicated. Hence,
the approach used here includes only two components in the composite
needs index\footnote{In contrast to the four of
  \citet{currie2010identifying}, which included two ``relative need''
  components obtained by weighting the IRSAD and the transport needs
  indexes by the population within the various needs groups for each
  area of interest. Current ABS reporting, however, does not allow the
  total number of people within one or more of the various needs groups
  to be identified at the SA1 level.} based on weighting both the IRSAD
index and the transport need index by the total population of each SA1.
These were then standardised and grouped, as per the six groups used by
\citet{currie2010identifying}\footnote{Very Low, Low, Below average,
  Above average, High and Very High.}.

\section{Results}\label{results}

\subsection{The gtfssupplyindex
package}\label{the-gtfssupplyindex-package}

Code developed to calculate SI scores is available as an R package on
github (see \citet{gtfssupplyindex_github}). Included in the package is
a vignette (see Appendix) that outlines the developed functions and
provides step-by-step calculations for the Mornington Peninsula Railway
as a worked example.

\subsection{Melbourne}\label{melbourne}

\subsubsection{Transport Supply
Categories}\label{transport-supply-categories}

Table \ref{tab:Greater_Melbourne_CCDs_SA1_table} summarises the
distribution of CCDs and SA1s across different Transport Supply
categories in 2006, 2016 and 2021. Figure
\ref{fig:Greater_Melbourne_population_2021_by_SA4} shows the spatial
distribution of Transport Supply by SA1 in 2021. Additional maps showing
Transport Supply by CCD and for 2016 are included in the Appendix.

\begin{table}

\caption{\label{tab:Greater_Melbourne_CCDs_SA1_table}Distribution of 2006, 2016 and 2021 Transport Supply to Melbourne CCDs (2006 boundaries), 2016 Transport Supply to Greater Melbourne (2016 SA1s) and 2021 Transport Supply to Greater Melbourne (2021 SA1s). Sources: 2006 values, Currie (2010); 2016 and 2021 values, authors' analysis}
\centering
\begin{tabular}[t]{l|r|r|r|r|r}
\hline
\multicolumn{1}{c|}{Transport} & \multicolumn{3}{c|}{CCDs} & \multicolumn{1}{c|}{2016 SA1s} & \multicolumn{1}{c}{2021 SA1s} \\
\cline{1-1} \cline{2-4} \cline{5-5} \cline{6-6}
Supply & 2006 & 2016 & 2021 & 2016 & 2021\\
\hline
Zero Supply & 3.2\%   (189) & 1.4\%    (86) & 1.3\%    (81) & 3.2\%    (326) & 4.3\%    (489)\\
\hline
Very Low & 22.5\% (1,314) & 23.5\% (1,485) & 23.3\% (1,474) & 23.0\%  (2,362) & 23.4\%  (2,692)\\
\hline
Low & 22.4\% (1,310) & 23.5\% (1,484) & 23.3\% (1,473) & 23.0\%  (2,362) & 23.4\%  (2,691)\\
\hline
Below average & 22.2\% (1,294) & 23.5\% (1,484) & 23.3\% (1,473) & 23.0\%  (2,362) & 23.4\%  (2,691)\\
\hline
Above average & 10.4\%   (608) & 9.4\%   (596) & 9.6\%   (608) & 9.3\%    (959) & 8.5\%    (975)\\
\hline
High & 9.2\%   (535) & 9.4\%   (595) & 9.6\%   (608) & 9.3\%    (959) & 8.5\%    (974)\\
\hline
Very High & 10.1\%   (589) & 9.4\%   (595) & 9.6\%   (608) & 9.3\%    (959) & 8.5\%    (975)\\
\hline
Total & 100.0\% (5,839) & 100.0\% (6,325) & 100.0\% (6,325) & 100.0\% (10,289) & 100.0\% (11,487)\\
\hline
\end{tabular}
\end{table}

There is a statistically significant difference in the shares of CCDs in
each category between 2006, 2016 and 2021
(\(\chi^2(12, N = 18489) = 87.45\), \(p < .001\))\footnote{Differences
  are also statistically significant when comparing 2006 and 2016
  (\(\chi^2(6, N = 12164) = 56.87\), \(p < .001\)) or 2006 and 2021
  \(\chi^2(6, N = 12164) = 59.15\), \(p < .001\)), but not between 2016
  and 2021 (\(\chi^2(6, N = 12650) = 0.67\), \(p = .995\)).}. Only 81
CCDs (1.3\%) have Zero Supply in 2021, compared to the 189 (3.2\%)
reported by \citet{currie2010identifying} for 2006. These shares by
CCDs, however, are only for 2006 extents of Melbourne. The ABS'
statistical boundary of ``Greater Melbourne'' now includes areas up to
around 30 kilometers further to the north. Figure
\ref{fig:Greater_Melbourne_population_2021_by_SA4} includes the 2006
boundary as an overlay, indicating that much of the new parts of Greater
Melbourne have Very Low or Zero Supply levels.

The difference between the share of areas of interest in each category
in 2006 (CCDs), 2016 and 2021 (SA1s) is also statistically significant
(\(\chi^2(12, N = 27615) = 58.86\), \(p < .001\))\footnote{Differences
  between 2006 and 2016, as a pair, are not statistically significant
  (\(\chi^2(6, N = 16128) = 8.83\), \(p = .183\)). Differences between
  2006 and 2021 are signficant (\(\chi^2(6, N = 17326) = 44.83\),
  \(p < .001\)) as are the differences between 2016 and 2021
  \(\chi^2(6, N = 21776) = 31.56\), \(p < .001\)).}. There is a greater
proportion of SA1s with Zero Supply in 2021 (4.3\%) than there were CCDs
in 2006 (3.25\%) or SA1s in 2016 (3.2\%). The share of SA1s with supply
below the average (i.e.~Zero, Very Low, Low or Below Average) is also
larger in 2021 (74.5\%) than the share of CCDs in 2006 (70.3\%) or share
of SA1s in 2016 (72.0\%).

\begin{table}

\caption{\label{tab:Greater_Melbourne_CCDs_SA1_population}Distribution of 2006, 2016 and 2021 Transport Supply to population in Melbourne. Sources: 2006 values, Currie (2010); 2016 and 2021 values, authors' analysis}
\centering
\begin{tabular}[t]{l|r|r|r}
\hline
Supply & 2006 & 2016 & 2021\\
\hline
Zero Supply & 2.5\%    (85,423) & 2.9\%   (131,619) & 3.8\%   (186,829)\\
\hline
Very Low & 23.6\%   (793,046) & 22.5\% (1,008,498) & 23.0\% (1,132,967)\\
\hline
Low & 25.7\%   (865,330) & 22.7\% (1,016,848) & 23.7\% (1,163,358)\\
\hline
Below average & 23.0\%   (774,521) & 22.3\% (1,000,290) & 23.6\% (1,159,783)\\
\hline
Above average & 9.6\%   (324,546) & 9.3\%   (418,614) & 8.7\%   (426,892)\\
\hline
High & 7.7\%   (260,411) & 9.6\%   (428,880) & 8.7\%   (425,779)\\
\hline
Very High & 7.8\%   (263,832) & 10.7\%   (480,469) & 8.6\%   (422,025)\\
\hline
Total & 100.0\% (3,367,109) & 100.0\% (4,485,218) & 100.0\% (4,917,633)\\
\hline
\end{tabular}
\end{table}

Table \ref{tab:Greater_Melbourne_CCDs_SA1_population} compares the share
of resident population in each transport supply category. Greater
Melbourne's population increased by 46\% between 2006 and 2021. However,
residents living in areas with Zero Supply rose by 119\% from 85,423
(2.5\% of the total population) in 2006 to 186,829 (3.8\%) in 2021.
Residents with Zero or Very Low Transport Supply rose by 50\% from
878,469 (26.1\%) in 2006 to 1,319,796 (26.8\%) in 2021, while those with
supply below the average (Zero, Very Low, Low or Below Average) rose by
45\% from 2,518,320 (74.8\%) in 2006 to 3,642,937 (74.1\%) in 2021

Between 2016 and 2021, Greater Melbourne's population increased by 10\%.
However, the number of residents in SA1s with Zero Supply rose by 42\%
(from 131,619 (2.9\%)). The number of residents with Zero or Very Low
Transport Supply rose by 16\% (from 1,140,117 (25.4\%)), while residents
with supply below the average (Zero, Very Low, Low or Below Average)
rose 15\% (from 3,157,255 (70.4\%)).

\begin{table}

\caption{\label{tab:Greater_Melbourne_population_2016_by_SA4}Greater Melbourne 2016: Share of population in each Transport Supply category for each SA4 region}
\centering
\fontsize{8}{10}\selectfont
\begin{tabular}[t]{>{\raggedright\arraybackslash}p{1.75cm}|>{\raggedleft\arraybackslash}p{1cm}|>{\raggedleft\arraybackslash}p{1cm}|>{\raggedleft\arraybackslash}p{1cm}|>{\raggedleft\arraybackslash}p{1cm}|>{\raggedleft\arraybackslash}p{1cm}|>{\raggedleft\arraybackslash}p{1cm}|>{\raggedleft\arraybackslash}p{1cm}|>{\raggedright\arraybackslash}p{1cm}|>{\raggedleft\arraybackslash}p{1cm}|>{\raggedleft\arraybackslash}p{1.25cm}}
\hline
Transport Supply & Inner & Inner East & Inner South & North East & North West & Outer East & South East & West & M'ton Pen. & Total\\
\hline
Zero Supply & 0.0\%       (0) & 0.0\%     (480) & 0.0\%   (1,604) & 0.4\%  (16,988) & 0.4\%  (17,655) & 0.3\%  (12,955) & 1.0\%  (44,757) & 0.3\%  (12,056) & 0.6\%  (25,124) & 2.9\%   (131,619)\\
\hline
Very Low & 0.1\%   (3,427) & 0.4\%  (18,454) & 0.6\%  (24,944) & 2.5\% (112,269) & 2.1\%  (94,853) & 4.3\% (190,890) & 4.8\% (215,217) & 4.2\% (186,665) & 3.6\% (161,779) & 22.5\% (1,008,498)\\
\hline
Low & 0.4\%  (18,018) & 0.9\%  (39,235) & 1.4\%  (60,833) & 2.7\% (119,608) & 2.4\% (107,693) & 3.0\% (135,247) & 5.0\% (224,097) & 5.4\% (242,438) & 1.6\%  (69,679) & 22.7\% (1,016,848)\\
\hline
Below average & 1.0\%  (42,950) & 2.3\% (105,168) & 2.9\% (128,014) & 2.9\% (132,008) & 2.2\%  (97,739) & 2.7\% (119,691) & 4.0\% (177,817) & 3.8\% (170,015) & 0.6\%  (26,888) & 22.3\% (1,000,290)\\
\hline
Above average & 1.0\%  (44,547) & 1.8\%  (80,002) & 1.8\%  (81,038) & 1.0\%  (46,965) & 0.6\%  (28,905) & 0.6\%  (25,188) & 1.2\%  (53,228) & 1.2\%  (54,895) & 0.1\%   (3,846) & 9.3\%   (418,614)\\
\hline
High & 2.9\% (129,533) & 1.7\%  (74,966) & 1.7\%  (74,617) & 1.0\%  (46,291) & 0.3\%  (14,464) & 0.3\%  (15,371) & 0.7\%  (33,365) & 0.9\%  (38,499) & 0.0\%   (1,774) & 9.6\%   (428,880)\\
\hline
Very High & 7.9\% (353,232) & 0.9\%  (41,416) & 0.7\%  (32,561) & 0.5\%  (21,197) & 0.0\%   (2,033) & 0.0\%     (314) & 0.2\%   (6,893) & 0.5\%  (22,823) & 0.0\%       (0) & 10.7\%   (480,469)\\
\hline
Total & 13.2\% (591,707) & 8.0\% (359,721) & 9.0\% (403,611) & 11.0\% (495,326) & 8.1\% (363,342) & 11.1\% (499,656) & 16.8\% (755,374) & 16.2\% (727,391) & 6.4\% (289,090) & 100.0\% (4,485,218)\\
\hline
\end{tabular}
\end{table}

\begin{table}

\caption{\label{tab:Greater_Melbourne_population_2021_by_SA4}Greater Melbourne 2021: Share of population in each Transport Supply category for each SA4 region}
\centering
\fontsize{8}{10}\selectfont
\begin{tabular}[t]{>{\raggedright\arraybackslash}p{1.75cm}|>{\raggedleft\arraybackslash}p{1cm}|>{\raggedleft\arraybackslash}p{1cm}|>{\raggedleft\arraybackslash}p{1cm}|>{\raggedleft\arraybackslash}p{1cm}|>{\raggedleft\arraybackslash}p{1cm}|>{\raggedleft\arraybackslash}p{1cm}|>{\raggedleft\arraybackslash}p{1cm}|>{\raggedright\arraybackslash}p{1cm}|>{\raggedleft\arraybackslash}p{1cm}|>{\raggedleft\arraybackslash}p{1.25cm}}
\hline
Transport Supply & Inner & Inner East & Inner South & North East & North West & Outer East & South East & West & M'ton Pen. & Total\\
\hline
Zero Supply & 0.0\%       (0) & 0.0\%     (478) & 0.0\%   (1,655) & 0.5\%  (26,563) & 0.5\%  (22,186) & 0.3\%  (14,125) & 1.1\%  (53,966) & 0.9\%  (43,898) & 0.5\%  (23,958) & 3.8\%   (186,829)\\
\hline
Very Low & 0.1\%   (4,169) & 0.4\%  (20,688) & 0.5\%  (22,483) & 2.7\% (130,715) & 2.3\% (110,814) & 4.1\% (200,810) & 5.1\% (250,684) & 4.5\% (221,337) & 3.5\% (171,267) & 23.0\% (1,132,967)\\
\hline
Low & 0.4\%  (20,329) & 0.9\%  (45,160) & 1.3\%  (63,802) & 2.7\% (132,001) & 2.5\% (123,314) & 3.2\% (155,603) & 5.4\% (265,995) & 5.9\% (289,518) & 1.4\%  (67,636) & 23.7\% (1,163,358)\\
\hline
Below average & 1.1\%  (54,918) & 2.7\% (133,305) & 3.0\% (148,585) & 3.1\% (151,240) & 2.6\% (129,918) & 2.3\% (114,658) & 4.2\% (204,093) & 3.8\% (184,466) & 0.8\%  (38,600) & 23.6\% (1,159,783)\\
\hline
Above average & 1.1\%  (56,422) & 1.4\%  (69,199) & 1.9\%  (92,875) & 0.9\%  (44,470) & 0.6\%  (27,140) & 0.5\%  (22,262) & 1.1\%  (53,328) & 1.1\%  (55,438) & 0.1\%   (5,758) & 8.7\%   (426,892)\\
\hline
High & 3.1\% (151,439) & 1.5\%  (73,687) & 1.4\%  (69,281) & 0.9\%  (43,985) & 0.2\%   (9,710) & 0.2\%  (10,905) & 0.5\%  (24,707) & 0.8\%  (41,100) & 0.0\%     (965) & 8.7\%   (425,779)\\
\hline
Very High & 6.7\% (329,654) & 0.6\%  (30,949) & 0.5\%  (23,906) & 0.2\%  (11,752) & 0.0\%   (1,285) & 0.0\%       (0) & 0.1\%   (7,308) & 0.3\%  (17,171) & 0.0\%       (0) & 8.6\%   (422,025)\\
\hline
Total & 12.5\% (616,931) & 7.6\% (373,466) & 8.6\% (422,587) & 11.0\% (540,726) & 8.6\% (424,367) & 10.5\% (518,363) & 17.5\% (860,081) & 17.3\% (852,928) & 6.3\% (308,184) & 100.0\% (4,917,633)\\
\hline
\end{tabular}
\end{table}

\begin{figure}
\centering
\includegraphics{ReynoldsCurrieQu2024_files/figure-latex/Greater_Melbourne_population_2021_by_SA4-1.pdf}
\caption{Greater Melbourne 2021: Transport Supply by SA1, overlayed with
suburban rail network (black), SA4 boundaries (blue) and 2006 Melbourne
extents (black)}
\end{figure}

Variations in the share of SA1s in each Transport Supply category across
SA4 zones are statistically significant in both 2016
(\(\chi^2(48, N = 10289) = 6195.85\), \(p < .001\)) and 2021
(\(\chi^2(48, N = 11487) = 7250.39\), \(p < .001\))\footnote{Tables
  showing Transport Supply by number of SA1s for each SA4 are included
  in the Appendix.}. Tables
\ref{tab:Greater_Melbourne_population_2016_by_SA4} and
\ref{tab:Greater_Melbourne_population_2021_by_SA4}\} show the population
in each supply category by SA4 for 2016 and 2021. In general, larger
shares of residents have lower than the average supply in outer areas of
Greater Melbourne\footnote{The North East, North West, Outer East, South
  East, West and Mornington Peninsula SA4 zones}. In 2016, 2,714,128
residents (60.5\% of the total population of Greater Melbourne) were
both living outside of the inner three SA4s and in SA1s with lower than
average supply. By 2021 this had increased to 3,127,365 residents
(63.6\% of the total population).

\subsubsection{Supply Index scores}\label{supply-index-scores}

\citet{currie2010identifying} reported an average SI score for CCDs in
2006 of 2,886.9 across Melbourne, with averages of 10,922.7, 2,694.9 and
764.3 across the inner, middle and outer suburbs, respectively. In
comparison, the average SI value found here for 2021 (using CCDs and
within the 2006 boundary) is 3,390, and for the inner, middle and outer
suburbs 12,275.7 (+12\%), 3,409.1 (+27\%) and 998.96 (+31\%)
respectively\footnote{The same grouping of Local Government Areas (LGAs)
  to inner, middle and outer suburb groups as used in
  \citet{currie2010identifying} was used for this analysis, although
  here the City of Stonnington was allocated entirely to the middle
  grouping, whereas \citet{currie2010identifying} allocated part of this
  LGA to the inner group.}. Results found here might not be exactly
comparable to those in \citet{currie2010identifying} due to geographic
inconsistencies\footnote{The \citet{currie2010identifying} results show
  5,839 total CCDs within Melbourne, whereas the shape files obtained
  for this analysis include 6,325 CCDs within the 2006 Melbourne
  boundary.}. Overall, however, the results appear to be consistent with
there having been increases in SI scores across all parts of Melbourne
since 2006, while the general pattern of higher average SI scores in
inner (and then middle) suburbs than in outer suburbs remains unchanged.

\begin{table}

\caption{\label{tab:Greater_Melbourne_2016_2021_ratio_map}Greater Melbourne: Share of 2021 population living in SA1s by change in transit service (2016 vs 2021) by SA4 region}
\centering
\fontsize{8}{10}\selectfont
\begin{tabular}[t]{>{\raggedright\arraybackslash}p{1.75cm}|>{\raggedleft\arraybackslash}p{1cm}|>{\raggedleft\arraybackslash}p{1cm}|>{\raggedleft\arraybackslash}p{1cm}|>{\raggedleft\arraybackslash}p{1cm}|>{\raggedleft\arraybackslash}p{1cm}|>{\raggedleft\arraybackslash}p{1cm}|>{\raggedleft\arraybackslash}p{1cm}|>{\raggedright\arraybackslash}p{1cm}|>{\raggedleft\arraybackslash}p{1cm}|>{\raggedleft\arraybackslash}p{1.25cm}}
\hline
Change & Inner & Inner East & Inner South & North East & North West & Outer East & South East & West & M'ton Pen. & Total\\
\hline
New service & 0.0\%       (0) & 0.0\%       (0) & 0.0\%       (0) & 0.2\%   (9,911) & 0.7\%  (36,817) & 0.0\%     (238) & 1.1\%  (53,254) & 0.8\%  (40,483) & 0.1\%   (3,038) & 2.9\%   (143,741)\\
\hline
Increased 30\% or more & 0.0\%   (1,843) & 0.0\%     (279) & 0.8\%  (37,932) & 0.9\%  (46,448) & 1.7\%  (83,007) & 0.1\%   (4,209) & 2.6\% (127,248) & 2.7\% (131,194) & 1.3\%  (65,724) & 10.1\%   (497,884)\\
\hline
Increased 10 to 30\% & 0.9\%  (45,197) & 0.1\%   (3,190) & 0.8\%  (41,577) & 0.8\%  (40,989) & 1.1\%  (56,013) & 0.5\%  (22,609) & 1.6\%  (77,391) & 2.1\% (101,767) & 0.4\%  (21,060) & 8.3\%   (409,793)\\
\hline
Increased 5 to 10\% & 1.5\%  (72,360) & 0.3\%  (13,018) & 1.1\%  (52,033) & 0.8\%  (37,258) & 0.6\%  (31,400) & 0.5\%  (26,666) & 1.1\%  (53,370) & 2.1\% (101,769) & 0.2\%  (10,149) & 8.1\%   (398,023)\\
\hline
Increased 3 to 5\% & 1.6\%  (79,047) & 0.5\%  (25,074) & 0.6\%  (30,595) & 1.2\%  (60,661) & 1.1\%  (55,445) & 0.5\%  (23,819) & 0.9\%  (45,773) & 1.8\%  (88,601) & 0.3\%  (14,666) & 8.6\%   (423,681)\\
\hline
Increased 1 to 3\% & 2.3\% (115,203) & 1.4\%  (67,357) & 1.2\%  (60,889) & 1.9\%  (92,896) & 0.9\%  (42,761) & 0.8\%  (39,263) & 1.5\%  (75,676) & 2.4\% (117,967) & 0.7\%  (32,811) & 13.1\%   (644,823)\\
\hline
Within 1\% & 2.6\% (128,666) & 3.8\% (187,974) & 2.0\%  (97,037) & 2.5\% (124,942) & 1.2\%  (57,899) & 5.6\% (274,423) & 5.4\% (266,787) & 3.0\% (146,702) & 2.2\% (109,151) & 28.3\% (1,393,581)\\
\hline
Reduced 1 to 3\% & 1.0\%  (49,729) & 0.8\%  (39,675) & 0.6\%  (27,179) & 0.7\%  (34,180) & 0.3\%  (13,986) & 0.8\%  (38,826) & 0.7\%  (34,125) & 0.4\%  (19,517) & 0.4\%  (18,446) & 5.6\%   (275,663)\\
\hline
Reduced 3 to 10\% & 1.8\%  (89,662) & 0.5\%  (25,930) & 0.9\%  (42,701) & 0.9\%  (44,261) & 0.2\%  (11,175) & 0.9\%  (43,143) & 0.5\%  (24,903) & 0.7\%  (33,329) & 0.2\%   (7,953) & 6.6\%   (323,057)\\
\hline
Reduced 10\% or more & 0.7\%  (35,224) & 0.2\%  (10,491) & 0.6\%  (30,989) & 0.5\%  (22,617) & 0.3\%  (13,678) & 0.6\%  (31,042) & 1.0\%  (47,588) & 0.6\%  (27,701) & 0.0\%   (1,228) & 4.5\%   (220,558)\\
\hline
Service withdrawn (magenta) & 0.0\%       (0) & 0.0\%       (0) & 0.0\%       (0) & 0.0\%   (1,777) & 0.0\%     (424) & 0.0\%   (1,270) & 0.0\%   (1,360) & 0.0\%   (1,015) & 0.0\%       (0) & 0.1\%     (5,846)\\
\hline
Never served (black) & 0.0\%       (0) & 0.0\%     (478) & 0.0\%   (1,655) & 0.5\%  (24,786) & 0.4\%  (21,762) & 0.3\%  (12,855) & 1.1\%  (52,606) & 0.9\%  (42,883) & 0.5\%  (23,958) & 3.7\%   (180,983)\\
\hline
Total & 12.5\% (616,931) & 7.6\% (373,466) & 8.6\% (422,587) & 11.0\% (540,726) & 8.6\% (424,367) & 10.5\% (518,363) & 17.5\% (860,081) & 17.3\% (852,928) & 6.3\% (308,184) & 100.0\% (4,917,633)\\
\hline
\end{tabular}
\end{table}

\begin{figure}
\centering
\includegraphics{ReynoldsCurrieQu2024_files/figure-latex/Greater_Melbourne_2016_2021_ratio_map-1.pdf}
\caption{Greater Melbourne: changes in SI between 2016 and 2021, by 2021
SA1, overlayed with 2006 Greater Melbourne boundary (black); 2021 SA4
boundaries (blue); and suburban railway lines (black)}
\end{figure}

Average SI scores for SA1s across all of Greater Melbourne increased
from 2,843.8 (2016) to 2,901.4 (2021, +2.0\%). There is a statistically
significant and strong correlation between the 2016 and 2021 SI scores
(\(r(11485) = 1.00\), \(p < .001\), \(r_s =.98\), \(p < .001\)). Figure
\ref{fig:Greater_Melbourne_2016_2021_ratio_map} shows changes in SI
scores between 2016 and 2021, categorised into those SA1s where service
levels have reduces (3 groups), stayed within 1\% (1 group), increased
(5 groups), and where transit services have been introduced, were
withdrawn or were not provided in either 2016 or 2021. There is a
statistically significant difference in the proportion of SA1s in each
service level change category across SA4 areas
(\(\chi^2(80) = 2818.27\), \(p < .001\))\footnote{Service withdrawn
  category removed due to low numbers. See Appendix for table}.

Table \ref{tab:Greater_Melbourne_2016_2021_ratio_map} shows changes in
SI score between 2016 and 2021 (column 1) by the 2021 population,
summarised for each SA4 zone (columns 2 to 10) and Greater Melbourne as
a whole (column 11). In 2021, slightly more than half of the population
across Greater Melbourne were living in SA1s where the SI scores were at
least one percent higher than they were in 2016 (51.2\%, 2,517,945
people). The share of population for whom the SI increased by at least
one percent was highest in the North-West (72.0\%, 305,443 people) and
West (68.2\%, 581,781 people) SA4s, but lowest in the Inner East
(29.2\%, 108,918 people) and Outer East (22.5\%, 116,804 people) SA4s.
Service levels stayed within one percent of the level in 2016 for
1,393,581 people living in Greater Melbourne in 2021 (28.3\%). The share
of population for whom the SI stayed within one percent was highest in
the Outer East (52.9\%, 274,423 people) and Inner East (50.3\%, 187,974
people). The SI was more than one percent lower than it had been in 2016
for 825,124 people (16.8\%) in 2021 across all of Greater Melbourne. The
share of the population living in areas where the SI had dropped by more
than one percent was largest in the Inner (28.3\%, 174,615 people),
Inner South (23.9\%, 100,869 people) and Outer East (22.0\%, 114,281
people) SA4s. Those living in SA1s that did not have transit service in
2021 and had not had service in 2016 either numbered 180,983 people
(3.7\%) in 2021 across all of Greater Melbourne. The share of the
population without transit in either 2016 or 2021 was highest amongst
the Mornington Peninsula (7.8\%, 23,958 people) and South East (6.1\%,
52,606, people) SA4s, whereas no one at all in the Inner SA4 was living
in a SA1 without at least a Very Low service level. Service had been
withdrawn since 2016 5,846 residents in 2021 across all of Greater
Melbourne (0.1\%). The share of the population living where transit
service had been withdrawn was largest in the Mornington Peninsula
(0.0\%, 23,958) and South East (6.1\%, 52,606) SA4s, whereas no one at
all in the Inner SA4 was living in a SA1 without at least a Very Low
service level.

\subsubsection{Social needs}\label{social-needs}

\begin{figure}
\centering
\includegraphics{ReynoldsCurrieQu2024_files/figure-latex/Greater_Melbourne_2021_social_needs-1.pdf}
\caption{Greater Melbourne 2021: Distribution of categories of composite
social need index scores, overlayed with: 2006 Greater Melbourne
boundary (black); SA4 boundaries (blue); and suburban railway lines
(dashed).}
\end{figure}

Figure \ref{fig:Greater_Melbourne_2021_social_needs} shows the
distribution of categories of social need index scores across Greater
Melbourne for 2021\footnote{A map for 2016 is included in the Appendix
  as Figure \ref{fig:Greater_Melbourne_2016_social_needs_appendix}}.
This is analogous to the 2006 map shown in Figure
\ref{fig:Currie_map_SI}(b) although, as discussed in the methodology
section above, it was not possible to exactly replicate the
\citet{currie2010identifying} social needs scoring approach due to
changes in the way census results are reported. In general, the spatial
grouping of different levels of social need appears less consistently
grouped in 2021 than in 2006, although this may be an artifact of the
shift to SA1s from CCDs\footnote{CCDs were originally devised to group
  the approximately 200 dwellings allocated to each individual census
  collector, whereas SA1s were introduced to be consistent in population
  (200 to 800 people, averaging 400) and character
  \citep{ABS_SA1s_CCDs}.}.

\subsubsection{Needs-gap}\label{needs-gap}

\begin{figure}
\centering
\includegraphics{ReynoldsCurrieQu2024_files/figure-latex/Greater_Melbourne_2021_needs_gap_scatterplot_figure-1.pdf}
\caption{Greater Melbourne 2021, SI and Combined Needs Index scores,
with SI scores \textless{} 10 rounded up to equal 10 to improve
clarity.}
\end{figure}

Figure \ref{fig:Greater_Melbourne_2021_needs_gap_scatterplot_figure}
shows social needs and SI scores for 2021\footnote{To improve the
  clarity of the figure, SI scores less than 10 have been adjusted to
  equal 10. As well, those SA1s with combinations of Zero or Very Low
  Transport Supply and (especially) high needs scores (more than 40); or
  Very High supply and Very Low needs scores (less than 4.5) have been
  labelled with their SA2 name, so as to given an indication of which
  suburbs of Melbourne are at each of the extremes. A similar plot for
  2016 is shown in the Appendix.}. This is analogous to the 2006 plot
shown in Figure 1(c). There is a significant, but only weakly positive
correlation between the SI and Combined Needs Index scores for both 2016
(\(r(11136) = .06\), \(p < .001\), \(r_s =.07\), \(p < .001\)) and 2021
(\(r(11136) = .06\), \(p < .001\), \(r_s =.07\), \(p < .001\))

Differences in the share of SA1s in each Transport Supply category
across different Combined Needs Index categories are statistically
significant in 2016 (\(\chi^2(30, N = 9964) = 264.26\), \(p < .001\) and
2021 (\(\chi^2(30, N = 11138) = 133.51\), \(p < .001\))\footnote{See
  Appendix for tables by SA1.}. Tables
\ref{tab:Greater_Melbourne_2016_needs_gap_population_table} and
\ref{tab:Greater_Melbourne_2021_needs_gap_population_table} compare the
populations within each Transport Supply and Combined Needs Index
grouping for 2016 and 2021. In 2016 301,743 people lived within SA1s
that have Zero or Very Low Transport Supply, but Very High social needs
(6.7\% of the total population). By 2021 this had increased to 333,887
people (+11\%, 6.7\% of the total population). These compare to the
139,004 people reported in \citet{currie2010identifying} for 2006 (4.2\%
of the population).

\begin{figure}

{\centering \subfloat[2016\label{fig:Greater_Melbourne_2016_needs_gap_map_figure-1}]{\includegraphics[width=0.5\linewidth]{ReynoldsCurrieQu2024_files/figure-latex/Greater_Melbourne_2016_needs_gap_map_figure-1} }\subfloat[2021\label{fig:Greater_Melbourne_2016_needs_gap_map_figure-2}]{\includegraphics[width=0.5\linewidth]{ReynoldsCurrieQu2024_files/figure-latex/Greater_Melbourne_2016_needs_gap_map_figure-2} }

}

\caption{Greater Melbourne Transport Supply groupings overlayed with SA1s with very high transport need areas with zero or very low public Transport Supply (red).}\label{fig:Greater_Melbourne_2016_needs_gap_map_figure}
\end{figure}

Figure \ref{fig:Greater_Melbourne_2016_needs_gap_map_figure} show SA1
zones in Greater Melbourne with Very High transport needs, but Very Low
or Zero Transport Supply for 2016 and 2021, which is analogous to the
2006 distribution shown in Figure \ref{fig:Currie_map_SI}(d). Comparison
suggests that areas with large gaps between social needs and Transport
Supply continue to be mostly in the outer areas of Melbourne, but such
comparisons appear complicated by the larger number of SA1s than CCDs.
Figure \ref{fig:Greater_Melbourne_2016_needs_gap_map_figure} appears to
show more (smaller) areas with Very High transport needs, but Very Low
or Zero Transport Supply. This includes areas in the West, North West,
North East and South East SA4s, whereas in 2006 these parts of Melbourne
did not appear to have large gaps between need and supply. As well the
south-west parts of the Morning Peninsula (around Blairgowrie) drop out
of the category of having the largest needs-gaps in 2016, but are back
in this category again in 2021.

\subsubsection{Needs-gap and service level
changes}\label{needs-gap-and-service-level-changes}

\begin{table}

\caption{\label{tab:Greater_Melbourne_2016_needs_gap_SA4_service_change}Greater Melbourne: 2016 residents living in areas with Very High needs but Very Low or Zero supply, by SA4 and change in SI by 2021}
\centering
\fontsize{8}{10}\selectfont
\begin{tabular}[t]{>{\raggedright\arraybackslash}p{2.5cm}|>{\raggedleft\arraybackslash}p{1cm}|>{\raggedleft\arraybackslash}p{1cm}|>{\raggedleft\arraybackslash}p{1cm}|>{\raggedleft\arraybackslash}p{1cm}|>{\raggedleft\arraybackslash}p{1cm}|>{\raggedleft\arraybackslash}p{1cm}|>{\raggedleft\arraybackslash}p{1cm}|>{\raggedright\arraybackslash}p{1cm}|>{\raggedleft\arraybackslash}p{1.25cm}}
\hline
Change & Inner & Inner South & North East & North West & Outer East & South East & West & M'ton Pen. & Total\\
\hline
New or 30\%+ & 0.0\%     (0) & 0.5\% (1,512) & 3.6\% (10,928) & 7.2\% (21,707) & 0.0\%      (0) & 15.0\% (45,117) & 10.8\% (32,552) & 1.8\%  (5,342) & 38.8\% (117,158)\\
\hline
Increased 1 to 30\% & 0.2\%   (568) & 0.2\%   (648) & 3.4\% (10,172) & 3.0\%  (8,973) & 1.9\%  (5,843) & 6.5\% (19,533) & 7.6\% (22,823) & 2.1\%  (6,188) & 24.8\%  (74,748)\\
\hline
Within 1\% & 0.2\%   (546) & 0.4\% (1,271) & 1.9\%  (5,639) & 0.3\%    (953) & 4.8\% (14,489) & 5.3\% (15,974) & 4.1\% (12,378) & 4.1\% (12,421) & 21.1\%  (63,671)\\
\hline
Reduced, withdrawn, never & 0.0\%     (0) & 0.8\% (2,473) & 4.3\% (13,064) & 0.6\%  (1,878) & 1.2\%  (3,619) & 1.9\%  (5,835) & 3.9\% (11,773) & 2.5\%  (7,524) & 15.3\%  (46,166)\\
\hline
Total & 0.4\% (1,114) & 2.0\% (5,904) & 13.2\% (39,803) & 11.1\% (33,511) & 7.9\% (23,951) & 28.7\% (86,459) & 26.4\% (79,526) & 10.4\% (31,475) & 100.0\% (301,743)\\
\hline
\end{tabular}
\end{table}

\begin{figure}
\includegraphics[width=1\linewidth]{ReynoldsCurrieQu2024_files/figure-latex/Greater_Melbourne_2016_needs_gap_SA4_service_change-1} \caption{Greater Melbourne: change in transit supply between 2016 and 2021 for SA1s with Very High needs but Very Low or Zero supply in 2016}\label{fig:Greater_Melbourne_2016_needs_gap_SA4_service_change}
\end{figure}

Figure \ref{fig:Greater_Melbourne_needs_gap_map_figure} shows the change
in transit supply between 2016 and 2021 for SA1s with Very High needs
but Very Low or Zero supply in 2016. There is a statistically
significant variation across the non-Inner SA4s\footnote{Categories of
  change in service level collapsed and the Inner, Inner East and Inner
  South SA4s removed to meet assumption of the chi-square test.}
\(\chi^2(15) = 92.59\), \(p < .001\). Table
\ref{tab:Greater_Melbourne_2016_needs_gap_SA4_service_change} shows how
much the transit service level changed for populations living in SA1s
with Very High needs but Very Low or Zero supply in 2016 by SA4. Service
levels had increased by 2021 for most of those living in Very High needs
but Very Low or Zero supply SA1s in 2016 in the North West (92\%, 30,680
people), South East (75\%, 64,650 people), the West (70\%, 55,375
people) and North East (53\%, 21,100 people). However, services had been
reduced or withdrawn, or were never present for 33\% of people living in
SA1s with Very High needs and Very Low or Zero supply in 2016 in the
North East (13,064people), 24\% of people living in SA1s with Very High
needs and Very Low or Zero supply in 2016 in the Mornington Peninsula
(7,524 people), 15\% in the West (11,773 people) and 15\% of people
living in SA1s with Very High needs and Very Low or Zero supply in 2016
in the Outer East (3,619 people).

\section{Discussion}\label{discussion}

This research was motivated by a lack of software tools to analyse gaps
between social needs for transport and the amount of transit provided,
and a need to better understand how spatial patterns might have changed
since the original development of this methodology in the early 2000s.
The needs-gaps approach was suggested to be ``substantially more useful
than the presentation of anecdotal evidence, which is the most common
means of identifying transport needs in local transport studies
throughout the world'' \citep{currie2010identifying}, yet this technique
does not appear to have been adopted in practice or received much
further attention from researchers. The research reported in this paper,
therefore, is important because it may help bridge gaps between research
(where the needs-gap approach and SI formulation was developed) and the
real world of transport planning, advocacy, professional practice and
politics (where decisions are made about where transit services are
needed or need improving).

As discussed in Section 2, there are many transit metrics available, but
these may be impossible to calculate independently or complicated to use
for the purposes of communicating with the public, decision-makers or
other stakeholders. The SI has the advantage of combining accessibility
and service frequency into a single measure, which is relatively simple
to understand and can even be calculated by hand for smaller networks.
The gtfstools R package developed in this research may make it possible
for practitioners to apply the SI calculation across even relatively
large transit networks, such as that of Greater Melbourne, with a GTFS
feed and some computer time. While the tool is not yet at a download,
point and click level of usability, it is open source and publicly
available and so may make assessing transit service levels and network
change proposals with respect to social needs for transport more widely
achievable.

The maps, graphs, tables and other outputs presented in this paper also
demonstrate the range of analysis that might be possible using this
package. Longitudinal comparison of transit service levels has been
shown in aggregate (Table
\ref{tab:Greater_Melbourne_CCDs_SA1_population}), by population across
sub-areas (in comparing Tables
\ref{tab:Greater_Melbourne_population_2016_by_SA4} and
\ref{tab:Greater_Melbourne_population_2021_by_SA4}) and mapped (Figure
\ref{fig:Greater_Melbourne_2016_2021_ratio_map}). Visual display of gaps
between social needs and Transport Supply how also been shown through
scatterplots (Figures
\ref{fig:Greater_Melbourne_2016_needs_gap_scatterplot} and
\ref{fig:Greater_Melbourne_2021_needs_gap_scatterplot}) and mapping
(Figures \ref{fig:Greater_Melbourne_2016_needs_gap_map} and
\ref{fig:Greater_Melbourne_2021_needs_gap_map}). Such outputs,
especially if produced at the spatial level of a neighbourhood, council
ward or electoral division, might provide an opportunity for advocates
and professionals within transport planning to identify and easily
demonstrate to the public, politician and other stakeholders which
spatial areas might benefit from increasing transit funding and where
funds might best be directed. The map shown in Figure
\ref{fig:Greater_Melbourne_2016_needs_gap_SA4_service_change}
demonstrates how service levels for those residents who had Very High
transport needs, but Very Low or Zero Supply in 2016 changed by 2021,
indicating that

Previous research looking at bus services in new developments includes
the \citet{delbosc2015impact} case study of the Selandra Rise planned
estate in the south-east of Melbourne. This study explored the impact of
the, then newly introduced, 798 bus route (Figure \ref{fig:Bus_798}. The
route was reported as having around 2,500 boardings per week and being
used by at least one person in 35 percent of households within the
Selandra Rise Estate, despite it not penetrating fully into the
development and only 20 percent of the estate's footprint being within
400 metres of the route. Findings suggested that: the route was
``performing an important `social transit' function''; that ``those who
do use the bus use it very frequently (88 percent (of riders) at least a
few days a week)''; ``Most bus riders were young and could not drive or
had no car available, making them quite reliant on the bus'', and that
it ``frees up the time of other household members who would have had to
provide lifts'' \citep[p.10]{delbosc2015impact}. Figure
\ref{fig:Selandra_rise} revisits the Selandra Rise Estate, showing
Transport Supply in 2016 and 2021, and changes in SI score for SA1s with
Very High social needs, but Very Low or Zero Transport Supply.

\begin{figure}

{\centering \includegraphics[width=1\linewidth]{graphics/Route798} 

}

\caption{Bus Route 798 as originally introduced in 2014 (Source: Delbosc et al (2015))}\label{fig:Bus_798}
\end{figure}

Various transit routes appear to have been added to the area surrounding
the Selandra Rise Estate between 2016 (Figure \ref{fig:Selandra_rise},
top-left), 2021 (top-right) and 2024 (bottom-right). After 2016 Route
798 was adjusted to penetrate into the estate, and Route 888 was added
to provide a north-south connection along Berwick-Cranbourne Road. This
appears to have been sufficient to shift some of the estate into the Low
Transport Supply category in 2021, up from Very Low category in 2016.
Figure \ref{fig:Selandra_rise}, bottom-left, however, indicates that
some of the surrounding SA1s, however, still fall into the Very High
needs and Very Low or Zero supply social needs-gap category, including
some SA1s that have over 700 residents but did not have a transit
service within 400 metres in either 2016 and 2021. That said, it appears
that Route 898 has been added between 2021 and 2024, providing an
additional east-west connection and coverage, suggesting that the
ultimate transit network for the area is not yet implemented.

What the information shown in the bottom-left and bottom-right of Figure
\ref{fig:Selandra_rise} might suggest, however, is that areas in the
vicinity of Casey Fields are underserved. Revising Route 795 so as to
service areas north of Ballarto Rd, perhaps instead of duplicating parts
of Route 796, might be one option to increase coverage and provide some
stops for residents living in areas that did not have an transit within
400 metres in either 2016 or 2021.

\section{Conclusions}\label{conclusions}

This paper has reported the development of a new R package containing
tools for identifying gaps between the social need for transport and the
service provided based on GTFS data. It has also demonstrated how this
package might be applied to the case of Greater Melbourne in 2016 and
2021, in the same manner in which \citet{currie2010identifying} applied
the analysis approach to Melbourne in 2006.

A motivation for this study was explore how spatial patterns related to
Transport Supply, need and gaps have changed in Melbourne since 2006. To
some extent the results indicate the challenges of making longitudinal
comparisons when there are changes in statistical geography. The shift
from CCDs to SA1s by the ABS appears to have allowed results to be
output here are a finer scale than in 2006, due to the generally smaller
size of SA1s and the way in which they are focused towards containing
consistent land use patterns within each zone. This contrasts to the
CCDs, which instead denoted the areas within which one census collector
operated. Unfortunately, this limits the extend to which the 2016 and
2021 results can be directly compared to those for 2006. However, it is
notable that the share of the population living in areas wtih Zero
Transport Supply grew with time from 2.5\% (85,423 residents) in 2006 to
2.9\% (131,619 residents) in 2016, and then again to 3.8\% (186,829
residents) in 2021. How much of the change between 2006 and 2016 relates
to the switch in statistical geography is unclear, but the difference
between 2016 and 2021 suggests that more people and a greater proportion
of the population are living in areas where transit stops are beyond
typical walking access distance as time has marched on.

Compared to the 8.2 percent (276,739) of Melbourne residents having very
high needs but zero, low or, very low Transport Supply in 2006 reported
by \citet{currie2010identifying}, this study found the equivalents to be
11.2\% (502,680 residents) in 2016 and 11.2\% ( 301,743 residents) in
2021. Again, the 2016 and 2021 values may not directly comparable to
those from 2006 because of the statistical geography changes. However,
these findings might again suggest that the situation is moving in the
wrong direction, at least in Greater Melbourne.

The extent to which the findings from this case study of Greater
Melbourne can be confidently generalised to other places (relating to
the duality criterion of case research) is a key question. Fortunately,
the widespread availability of GTFS datasets together with the new
gtfssupplyindex package means that it will be easier to analyse other
cases. Exploring social needs-gaps in other Australian cities is an
immediate direction for future efforts associated with this program of
research. In the meantime, it would seem prudent to be cautious as to
making assumptions about whether shifts in Melbourne are representative
of everywhere else. The almost 50 percent growth in population in
Greater Melbourne between 2006 (3.4m) and 2021 (4.9m) is towards the
higher end of that experienced in other places (REFERENCE), although
low-density development patterns that provide similar challenges for
transit planning appear common both within and beyond Australia. The
balance of spending between major projects (e.g.~highways, rail-based
transit) and introducing or increasing the levels of service of transit
provided primarily to meet social needs for transport will be different
in Melbourne to that in other locations. However, with a new subway
connection through the CBD and a second freeway connection across the
largest river currently under construction, and various other rail and
road mega-projects either in planning or recently completed, Melbourne
might be an outlier as far as the share of government spending going
towards transport that is not focused on providing basic coverage for
those who cannot otherwise drive or increasing service levels for those
with the greatest social need for transit. The challenges in building
business cases, increasing subsidies and otherwise introducing or
increasing transit service would appear likely to be similar in
Melbourne as to other places. Hence, it might be confidentially expected
that the overall patterns, of the largest gaps between the social need
for transport and the transit supplied being in outer suburbs and away
from suburban railway lines, might be reasonably expected in other
cities.

As well as making comparisons of Melbourne across 2006, 2016 and 2021,
this paper has expanded on the analysis approach developed in
\citet{Currie2003Hobart}, \citet{Currie2004Gap},
\citet{Currie2007Identifying} and \citet{currie2010identifying}. The
mapping of changes in SI scores (Figure
\ref{fig:Greater_Melbourne_2016_2021_ratio_map}) and how these changes
relate to transit routes and areas with large needs-gaps at a
neighbourhood (Figure \ref{fig:Selandra_rise}) might be of use to
transport planners, advocates, politicians and other decision-makers,
and others seeking to build legitimacy for the improvement or
introduction of basic transit service levels. It is heartening that the
Selandra Rise Estate, where \citet{delbosc2015impact} found that what
transit was initially implemented provide a very important social
function and that there was a community desire for greater levels of
service, now has more routes and higher service levels. However, Figure
\ref{fig:Selandra_rise} shows that there were similarly
under-or-never-served communities nearby in 2021, suggesting that the
timely introduction of transit service to new development areas
continues to be a challenge in parts of Melbourne. Further research
might seek to confirm whether similar issues are evident in outer and
new development areas of other cities, although this would seem likely
to be the case in many other places.

Overall, the research reported in this paper may help researchers,
practitioners, advocates, other stakeholders and decision-makers obtain
information about spatial areas that have gaps between the social need
for transport and the transit supplied. \citet{currie2010identifying}
suggested that using the Supply Index and needs-gap analysis techniques
would be much preferable to anecdotal or other informal inputs to
decision-making related to providing transit to meet the basic social
needs for transport of communities. The software package and analysis
approaches reported in this paper, enabling GTFS data to be directly
used as an input, will hopefully allow the social needs-gap approach to
be used more easily and widely.

\section*{References}\label{references}
\addcontentsline{toc}{section}{References}

\section{Appendix}\label{appendix}

\begin{figure}

{\centering \subfloat[Melbourne (2006 extents), Transport Supply by CCD for the week starting the date of the 2016 census, overlayed with suburban railway lines (black) and inner, middle and outer suburban boundary\label{fig:Greater_Melbourne_CCD_2016_appendix-1}]{\includegraphics[width=0.5\linewidth]{ReynoldsCurrieQu2024_files/figure-latex/Greater_Melbourne_CCD_2016_appendix-1} }\subfloat[Melbourne (2006 extents), Transport Supply by CCD for the week starting the date of the 2021 census, overlayed with suburban railway lines (black) and inner, middle and outer suburban boundary\label{fig:Greater_Melbourne_CCD_2016_appendix-2}]{\includegraphics[width=0.5\linewidth]{ReynoldsCurrieQu2024_files/figure-latex/Greater_Melbourne_CCD_2016_appendix-2} }

}

\caption{Transport Supply Categories. Source: Currie (2010)}\label{fig:Greater_Melbourne_CCD_2016_appendix}
\end{figure}

\begin{table}

\caption{\label{tab:Greater_Melbourne_SA1_2016_by_SA4}Greater Melbourne 2016: SA1s in each Transport Supply category by SA4}
\centering
\fontsize{8}{10}\selectfont
\begin{tabular}[t]{>{\raggedright\arraybackslash}p{1.75cm}|>{\raggedleft\arraybackslash}p{1cm}|>{\raggedleft\arraybackslash}p{1cm}|>{\raggedleft\arraybackslash}p{1cm}|>{\raggedleft\arraybackslash}p{1cm}|>{\raggedleft\arraybackslash}p{1cm}|>{\raggedleft\arraybackslash}p{1cm}|>{\raggedleft\arraybackslash}p{1cm}|>{\raggedright\arraybackslash}p{1cm}|>{\raggedleft\arraybackslash}p{1cm}|>{\raggedleft\arraybackslash}p{1.25cm}}
\hline
Transport Supply & Inner & Inner East & Inner South & North East & North West & Outer East & South East & West & M'ton Pen. & Total\\
\hline
Zero Supply & 0.0\%     (0) & 0.0\%   (1) & 0.0\%   (5) & 0.5\%    (48) & 0.4\%  (43) & 0.4\%    (42) & 1.0\%   (101) & 0.2\%    (23) & 0.6\%  (63) & 3.2\%    (326)\\
\hline
Very Low & 0.1\%    (10) & 0.5\%  (50) & 0.7\%  (70) & 2.5\%   (257) & 2.1\% (215) & 5.0\%   (511) & 4.6\%   (473) & 3.7\%   (377) & 3.9\% (399) & 23.0\%  (2,362)\\
\hline
Low & 0.4\%    (44) & 0.9\%  (95) & 1.4\% (142) & 2.8\%   (288) & 2.4\% (243) & 3.2\%   (329) & 5.0\%   (518) & 5.3\%   (541) & 1.6\% (162) & 23.0\%  (2,362)\\
\hline
Below average & 1.0\%   (105) & 2.4\% (249) & 2.9\% (298) & 3.1\%   (319) & 2.2\% (231) & 2.8\%   (284) & 4.0\%   (411) & 3.9\%   (399) & 0.6\%  (66) & 23.0\%  (2,362)\\
\hline
Above average & 1.1\%   (111) & 1.8\% (186) & 1.8\% (184) & 1.0\%   (103) & 0.7\%  (70) & 0.6\%    (60) & 1.1\%   (114) & 1.2\%   (122) & 0.1\%   (9) & 9.3\%    (959)\\
\hline
High & 2.7\%   (279) & 1.7\% (175) & 1.6\% (168) & 1.0\%   (107) & 0.3\%  (31) & 0.4\%    (39) & 0.7\%    (70) & 0.8\%    (86) & 0.0\%   (4) & 9.3\%    (959)\\
\hline
Very High & 6.7\%   (685) & 0.8\%  (86) & 0.7\%  (75) & 0.4\%    (41) & 0.1\%   (7) & 0.0\%     (1) & 0.2\%    (16) & 0.5\%    (48) & 0.0\%   (0) & 9.3\%    (959)\\
\hline
Total & 12.0\% (1,234) & 8.2\% (842) & 9.2\% (942) & 11.3\% (1,163) & 8.2\% (840) & 12.3\% (1,266) & 16.6\% (1,703) & 15.5\% (1,596) & 6.8\% (703) & 100.0\% (10,289)\\
\hline
\end{tabular}
\end{table}

\begin{figure}
\centering
\includegraphics{ReynoldsCurrieQu2024_files/figure-latex/Greater_Melbourne_SA12016_plot_appendix-1.pdf}
\caption{Greater Melbourne 2016: Transport Supply by SA1, overlayed with
suburban rail network (black), SA4 boundaries (blue) and 2006 Melbourne
extents (black)}
\end{figure}

\begin{table}

\caption{\label{tab:Greater_Melbourne_SA1_2021_by_SA4}Greater Melbourne 2016: SA1s in each Transport Supply category by SA4}
\centering
\fontsize{8}{10}\selectfont
\begin{tabular}[t]{>{\raggedright\arraybackslash}p{1.75cm}|>{\raggedleft\arraybackslash}p{1cm}|>{\raggedleft\arraybackslash}p{1cm}|>{\raggedleft\arraybackslash}p{1cm}|>{\raggedleft\arraybackslash}p{1cm}|>{\raggedleft\arraybackslash}p{1cm}|>{\raggedleft\arraybackslash}p{1cm}|>{\raggedleft\arraybackslash}p{1cm}|>{\raggedright\arraybackslash}p{1cm}|>{\raggedleft\arraybackslash}p{1cm}|>{\raggedleft\arraybackslash}p{1.25cm}}
\hline
Transport Supply & Inner & Inner East & Inner South & North East & North West & Outer East & South East & West & M'ton Pen. & Total\\
\hline
Zero Supply & 0.0\%     (0) & 0.0\%   (1) & 0.0\%   (5) & 0.6\%    (71) & 0.5\%  (55) & 0.4\%    (44) & 1.2\%   (142) & 1.0\%   (112) & 0.5\%  (59) & 4.3\%    (489)\\
\hline
Very Low & 0.1\%    (13) & 0.5\%  (54) & 0.5\%  (63) & 2.7\%   (307) & 2.2\% (254) & 4.6\%   (523) & 5.0\%   (576) & 4.4\%   (504) & 3.5\% (398) & 23.4\%  (2,692)\\
\hline
Low & 0.4\%    (48) & 0.9\% (108) & 1.3\% (147) & 2.8\%   (322) & 2.4\% (279) & 3.2\%   (368) & 5.3\%   (608) & 5.7\%   (653) & 1.4\% (158) & 23.4\%  (2,691)\\
\hline
Below average & 1.1\%   (130) & 2.7\% (310) & 2.9\% (333) & 3.2\%   (363) & 2.6\% (297) & 2.3\%   (261) & 4.1\%   (470) & 3.8\%   (437) & 0.8\%  (90) & 23.4\%  (2,691)\\
\hline
Above average & 1.2\%   (137) & 1.4\% (160) & 1.8\% (210) & 0.9\%   (101) & 0.5\%  (63) & 0.4\%    (49) & 1.0\%   (114) & 1.1\%   (128) & 0.1\%  (13) & 8.5\%    (975)\\
\hline
High & 3.0\%   (345) & 1.5\% (172) & 1.4\% (161) & 0.8\%    (95) & 0.2\%  (22) & 0.3\%    (29) & 0.5\%    (55) & 0.8\%    (93) & 0.0\%   (2) & 8.5\%    (974)\\
\hline
Very High & 6.6\%   (763) & 0.6\%  (65) & 0.5\%  (57) & 0.2\%    (26) & 0.0\%   (5) & 0.0\%     (0) & 0.2\%    (19) & 0.3\%    (40) & 0.0\%   (0) & 8.5\%    (975)\\
\hline
Total & 12.5\% (1,436) & 7.6\% (870) & 8.5\% (976) & 11.2\% (1,285) & 8.5\% (975) & 11.1\% (1,274) & 17.3\% (1,984) & 17.1\% (1,967) & 6.3\% (720) & 100.0\% (11,487)\\
\hline
\end{tabular}
\end{table}

\begin{table}

\caption{\label{tab:Greater_Melbourne_2016_2021_ratio_SA1_table}Greater Melbourne: Share of 2021 SA1s by change in transit service (2016 vs 2021) by SA4 region}
\centering
\fontsize{8}{10}\selectfont
\begin{tabular}[t]{>{\raggedright\arraybackslash}p{1.75cm}|>{\raggedleft\arraybackslash}p{1cm}|>{\raggedleft\arraybackslash}p{1cm}|>{\raggedleft\arraybackslash}p{1cm}|>{\raggedleft\arraybackslash}p{1cm}|>{\raggedleft\arraybackslash}p{1cm}|>{\raggedleft\arraybackslash}p{1cm}|>{\raggedleft\arraybackslash}p{1cm}|>{\raggedright\arraybackslash}p{1cm}|>{\raggedleft\arraybackslash}p{1cm}|>{\raggedleft\arraybackslash}p{1.25cm}}
\hline
Change & Inner & Inner East & Inner South & North East & North West & Outer East & South East & West & M'ton Pen. & Total\\
\hline
New service & 0.0\%     (0) & 0.0\%   (0) & 0.0\%   (0) & 0.2\%    (20) & 0.7\%  (79) & 0.0\%     (1) & 1.0\%   (113) & 0.8\%    (87) & 0.1\%   (6) & 2.7\%    (306)\\
\hline
Increased 30\% or more & 0.0\%     (5) & 0.0\%   (1) & 0.7\%  (83) & 0.9\%   (107) & 1.5\% (173) & 0.1\%    (10) & 2.6\%   (293) & 2.5\%   (290) & 1.3\% (154) & 9.7\%  (1,116)\\
\hline
Increased 10 to 30\% & 0.9\%   (104) & 0.1\%   (7) & 0.9\%  (99) & 0.8\%    (95) & 1.2\% (133) & 0.5\%    (54) & 1.5\%   (173) & 2.0\%   (231) & 0.4\%  (49) & 8.2\%    (945)\\
\hline
Increased 5 to 10\% & 1.4\%   (166) & 0.2\%  (28) & 1.0\% (120) & 0.8\%    (90) & 0.7\%  (75) & 0.6\%    (69) & 1.0\%   (120) & 2.0\%   (233) & 0.2\%  (23) & 8.0\%    (924)\\
\hline
Increased 3 to 5\% & 1.6\%   (185) & 0.5\%  (58) & 0.6\%  (71) & 1.3\%   (147) & 1.1\% (122) & 0.5\%    (63) & 0.9\%   (104) & 1.8\%   (210) & 0.3\%  (32) & 8.6\%    (992)\\
\hline
Increased 1 to 3\% & 2.4\%   (281) & 1.3\% (152) & 1.2\% (142) & 1.9\%   (214) & 0.9\% (107) & 0.9\%   (101) & 1.5\%   (176) & 2.4\%   (271) & 0.7\%  (76) & 13.2\%  (1,520)\\
\hline
Within 1\% & 2.5\%   (286) & 3.9\% (444) & 1.9\% (223) & 2.6\%   (295) & 1.2\% (138) & 5.8\%   (661) & 5.4\%   (620) & 3.0\%   (347) & 2.3\% (261) & 28.5\%  (3,275)\\
\hline
Reduced 1 to 3\% & 1.0\%   (114) & 0.8\%  (94) & 0.5\%  (63) & 0.8\%    (89) & 0.3\%  (32) & 0.8\%    (92) & 0.7\%    (77) & 0.4\%    (44) & 0.4\%  (42) & 5.6\%    (647)\\
\hline
Reduced 3 to 10\% & 1.8\%   (212) & 0.5\%  (61) & 0.9\% (100) & 0.9\%   (105) & 0.2\%  (28) & 0.9\%   (102) & 0.5\%    (58) & 0.7\%    (79) & 0.1\%  (15) & 6.6\%    (760)\\
\hline
Reduced 10\% or more & 0.7\%    (83) & 0.2\%  (24) & 0.6\%  (70) & 0.5\%    (52) & 0.3\%  (33) & 0.7\%    (77) & 0.9\%   (108) & 0.5\%    (63) & 0.0\%   (3) & 4.5\%    (513)\\
\hline
Service withdrawn (magenta) & 0.0\%     (0) & 0.0\%   (0) & 0.0\%   (0) & 0.0\%     (4) & 0.0\%   (1) & 0.0\%     (3) & 0.0\%     (3) & 0.0\%     (4) & 0.0\%   (0) & 0.1\%     (15)\\
\hline
Never served (black) & 0.0\%     (0) & 0.0\%   (1) & 0.0\%   (5) & 0.6\%    (67) & 0.5\%  (54) & 0.4\%    (41) & 1.2\%   (139) & 0.9\%   (108) & 0.5\%  (59) & 4.1\%    (474)\\
\hline
Total & 12.5\% (1,436) & 7.6\% (870) & 8.5\% (976) & 11.2\% (1,285) & 8.5\% (975) & 11.1\% (1,274) & 17.3\% (1,984) & 17.1\% (1,967) & 6.3\% (720) & 100.0\% (11,487)\\
\hline
\end{tabular}
\end{table}

\begin{figure}
\includegraphics[width=0.9\linewidth]{ReynoldsCurrieQu2024_files/figure-latex/Greater_Melbourne_2016_social_needs_appendix-1} \caption{Distribution of categories of composite social need index scores in 2016 (left) and 2021 (right), overlayed with: 2006 Greater Melbourne boundary (black); middle/outer and inner/middle suburb boundaries (grey); and suburban railway lines (dashed).}\label{fig:Greater_Melbourne_2016_social_needs_appendix}
\end{figure}

\begin{table}

\caption{\label{tab:Greater_Melbourne_2016_needs_gap_zones}Greater Melbourne 2016, SA1s within each SI and Combined Needs Index grouping}
\centering
\fontsize{7}{9}\selectfont
\begin{tabular}[t]{l|r|r|r|r|r|r|r}
\hline
transit\_supply & Very Low & Low & Below average & Above average & High & Very High & Total\\
\hline
Zero Supply & 4.8\%    (91) & 3.5\%    (66) & 2.5\%    (47) & 2.4\%    (34) & 2.0\%    (28) & 3.2\%    (46) & 3.1\%   (312)\\
\hline
Very Low & 26.1\%   (497) & 23.2\%   (442) & 20.9\%   (397) & 20.4\%   (290) & 19.8\%   (281) & 22.5\%   (319) & 22.3\% (2,226)\\
\hline
Low & 24.1\%   (458) & 24.5\%   (467) & 24.4\%   (464) & 22.3\%   (317) & 21.7\%   (308) & 19.7\%   (279) & 23.0\% (2,293)\\
\hline
Below average & 24.9\%   (473) & 24.5\%   (467) & 24.1\%   (459) & 23.8\%   (338) & 23.2\%   (329) & 17.6\%   (249) & 23.2\% (2,315)\\
\hline
Above average & 7.4\%   (140) & 9.2\%   (175) & 10.2\%   (194) & 10.5\%   (149) & 10.6\%   (150) & 9.2\%   (131) & 9.4\%   (939)\\
\hline
High & 6.5\%   (123) & 7.5\%   (142) & 10.7\%   (203) & 10.2\%   (145) & 12.0\%   (170) & 10.9\%   (155) & 9.4\%   (938)\\
\hline
Very High & 6.4\%   (121) & 7.6\%   (144) & 7.3\%   (139) & 10.3\%   (146) & 10.7\%   (152) & 16.9\%   (239) & 9.4\%   (941)\\
\hline
Total & 100.0\% (1,903) & 100.0\% (1,903) & 100.0\% (1,903) & 100.0\% (1,419) & 100.0\% (1,418) & 100.0\% (1,418) & 100.0\% (9,964)\\
\hline
\end{tabular}
\end{table}

\begin{table}

\caption{\label{tab:Greater_Melbourne_2021_needs_gap_zones}Greater Melbourne 2021, SA1s within each SI and Combined Needs Index grouping}
\centering
\fontsize{8}{10}\selectfont
\begin{tabular}[t]{>{\raggedright\arraybackslash}p{2.0cm}|>{\raggedleft\arraybackslash}p{1.25cm}|>{\raggedleft\arraybackslash}p{1.25cm}|>{\raggedleft\arraybackslash}p{1.25cm}|>{\raggedleft\arraybackslash}p{1.25cm}|>{\raggedleft\arraybackslash}p{1.25cm}|>{\raggedleft\arraybackslash}p{1.25cm}|>{\raggedleft\arraybackslash}p{1.5cm}}
\hline
\multicolumn{1}{c|}{ } & \multicolumn{6}{c|}{Combined Needs Index Category} & \multicolumn{1}{c}{ } \\
\cline{2-7}
transit\_supply & Very Low & Low & Below average & Above average & High & Very High & Total\\
\hline
Zero Supply & 6.4\%   (130) & 4.7\%    (95) & 3.9\%    (79) & 2.9\%    (49) & 3.0\%    (51) & 3.6\%    (61) & 4.2\%    (465)\\
\hline
Very Low & 25.6\%   (521) & 22.2\%   (452) & 21.8\%   (444) & 21.0\%   (352) & 22.0\%   (369) & 24.3\%   (408) & 22.9\%  (2,546)\\
\hline
Low & 23.5\%   (479) & 25.3\%   (514) & 24.6\%   (500) & 24.0\%   (403) & 22.4\%   (376) & 20.6\%   (346) & 23.5\%  (2,618)\\
\hline
Below average & 23.7\%   (483) & 23.7\%   (482) & 24.0\%   (488) & 25.6\%   (430) & 24.6\%   (412) & 20.8\%   (349) & 23.7\%  (2,644)\\
\hline
Above average & 6.9\%   (140) & 8.1\%   (165) & 9.2\%   (188) & 9.4\%   (157) & 9.1\%   (152) & 9.2\%   (154) & 8.6\%    (956)\\
\hline
High & 5.9\%   (120) & 8.0\%   (162) & 9.5\%   (193) & 9.2\%   (154) & 9.7\%   (162) & 9.8\%   (165) & 8.6\%    (956)\\
\hline
Very High & 8.0\%   (163) & 8.1\%   (165) & 7.0\%   (143) & 7.9\%   (133) & 9.2\%   (155) & 11.6\%   (194) & 8.6\%    (953)\\
\hline
Total & 100.0\% (2,036) & 100.0\% (2,035) & 100.0\% (2,035) & 100.0\% (1,678) & 100.0\% (1,677) & 100.0\% (1,677) & 100.0\% (11,138)\\
\hline
\end{tabular}
\end{table}

\begin{figure}
\centering
\includegraphics{ReynoldsCurrieQu2024_files/figure-latex/Greater_Melbourne_2016_needs_gap_scatterplot_figure-1.pdf}
\caption{Greater Melbourne 2016, SI and Combined Needs Index scores,
with SI scores \textless{} 10 rounded up to equal 10.}
\end{figure}

```

\begin{table}

\caption{\label{tab:Greater_Melbourne_2016_needs_gap_SA4_service_change}Greater Melbourne: SA1s with Very High needs but Very Low or Zero supply in 2016, by SA4 and change in SI by 2021}
\centering
\fontsize{8}{10}\selectfont
\begin{tabular}[t]{>{\raggedright\arraybackslash}p{2.5cm}|>{\raggedleft\arraybackslash}p{1cm}|>{\raggedleft\arraybackslash}p{1cm}|>{\raggedleft\arraybackslash}p{1cm}|>{\raggedleft\arraybackslash}p{1cm}|>{\raggedleft\arraybackslash}p{1cm}|>{\raggedleft\arraybackslash}p{1cm}|>{\raggedleft\arraybackslash}p{1cm}|>{\raggedleft\arraybackslash}p{1cm}|>{\raggedleft\arraybackslash}p{1cm}|>{\raggedleft\arraybackslash}p{1cm}}
\hline
Change & Inner & Inner South & Inner East & North East & North West & Outer East & South East & West & M'ton Pen. & Total\\
\hline
New or 30\%+ & 1.6\%  (6) & 6.6\% (24) & 14.5\%  (53) & 0.0\%  (0) & 5.5\% (20) & 3.0\% (11) & 0.5\% (2) & 0.0\% (0) & 0.0\% (0) & 31.8\% (116)\\
\hline
Increased 1 to 30\% & 2.5\%  (9) & 8.2\% (30) & 6.8\%  (25) & 2.5\%  (9) & 3.3\% (12) & 2.7\% (10) & 0.3\% (1) & 0.0\% (0) & 0.3\% (1) & 26.6\%  (97)\\
\hline
Within 1\% & 5.2\% (19) & 4.7\% (17) & 6.0\%  (22) & 6.3\% (23) & 0.3\%  (1) & 1.6\%  (6) & 0.5\% (2) & 0.0\% (0) & 0.3\% (1) & 24.9\%  (91)\\
\hline
Reduced, withdrawn, never & 2.7\% (10) & 3.6\% (13) & 2.5\%   (9) & 1.6\%  (6) & 0.8\%  (3) & 4.4\% (16) & 1.1\% (4) & 0.0\% (0) & 0.0\% (0) & 16.7\%  (61)\\
\hline
Total & 12.1\% (44) & 23.0\% (84) & 29.9\% (109) & 10.4\% (38) & 9.9\% (36) & 11.8\% (43) & 2.5\% (9) & 0.0\% (0) & 0.5\% (2) & 100.0\% (365)\\
\hline
\end{tabular}
\end{table}

\bibliography{References.bib, packages.bib}


\end{document}
