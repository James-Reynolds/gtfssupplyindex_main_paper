\documentclass[preprint, 3p,
authoryear]{elsarticle} %review=doublespace preprint=single 5p=2 column
%%% Begin My package additions %%%%%%%%%%%%%%%%%%%

\usepackage[hyphens]{url}

  \journal{Transport Geography?} % Sets Journal name

\usepackage{graphicx}
%%%%%%%%%%%%%%%% end my additions to header

\usepackage[T1]{fontenc}
\usepackage{lmodern}
\usepackage{amssymb,amsmath}
% TODO: Currently lineno needs to be loaded after amsmath because of conflict
% https://github.com/latex-lineno/lineno/issues/5
\usepackage{lineno} % add
\usepackage{ifxetex,ifluatex}
\usepackage{fixltx2e} % provides \textsubscript
% use upquote if available, for straight quotes in verbatim environments
\IfFileExists{upquote.sty}{\usepackage{upquote}}{}
\ifnum 0\ifxetex 1\fi\ifluatex 1\fi=0 % if pdftex
  \usepackage[utf8]{inputenc}
\else % if luatex or xelatex
  \usepackage{fontspec}
  \ifxetex
    \usepackage{xltxtra,xunicode}
  \fi
  \defaultfontfeatures{Mapping=tex-text,Scale=MatchLowercase}
  \newcommand{\euro}{€}
\fi
% use microtype if available
\IfFileExists{microtype.sty}{\usepackage{microtype}}{}
\usepackage[]{natbib}
\bibliographystyle{plainnat}

\usepackage{graphicx}
\ifxetex
  \usepackage[setpagesize=false, % page size defined by xetex
              unicode=false, % unicode breaks when used with xetex
              xetex]{hyperref}
\else
  \usepackage[unicode=true]{hyperref}
\fi
\hypersetup{breaklinks=true,
            bookmarks=true,
            pdfauthor={},
            pdftitle={Leveraging GTFS to explore spatial patterns in transit supply with respect to social needs},
            colorlinks=false,
            urlcolor=blue,
            linkcolor=magenta,
            pdfborder={0 0 0}}

\setcounter{secnumdepth}{5}
% Pandoc toggle for numbering sections (defaults to be off)


% tightlist command for lists without linebreak
\providecommand{\tightlist}{%
  \setlength{\itemsep}{0pt}\setlength{\parskip}{0pt}}




\usepackage{subfig}
\usepackage{booktabs}
\usepackage{longtable}
\usepackage{array}
\usepackage{multirow}
\usepackage{wrapfig}
\usepackage{float}
\usepackage{colortbl}
\usepackage{pdflscape}
\usepackage{tabu}
\usepackage{threeparttable}
\usepackage{threeparttablex}
\usepackage[normalem]{ulem}
\usepackage{makecell}
\usepackage{xcolor}



\begin{document}


\begin{frontmatter}

  \title{Leveraging GTFS to explore spatial patterns in transit supply
with respect to social needs}
    \author[Public Transport Research Group (PTRG)]{James Reynolds%
  %
  \fnref{1}}
   \ead{james.reynolds@monash.edu} 
    \author[Public Transport Research Group (PTRG)]{Yanda Qu%
  %
  \fnref{2}}
   \ead{yanda.qu@monash.edu} 
    \author[Public Transport Research Group (PTRG)]{Graham Currie%
  \corref{cor1}%
  \fnref{3}}
   \ead{graham.currie@monash.edu} 
      \affiliation[Public Transport Research Group (PTRG)]{
    organization={Public Transport Research Group (PTRG), Institute of
Transport Studies, Department of Civil Engineering Engineering, Monash
University},addressline={Clayton
Campus},city={Melbourne},postcode={3800},state={Victoria},country={Australia},}
    \cortext[cor1]{Corresponding author}
    \fntext[1]{Research Fellow}
    \fntext[2]{PhD Student}
    \fntext[3]{Professor}
  
  \begin{abstract}
  This is the abstract.

  It consists of two paragraphs.
  \end{abstract}
    \begin{keyword}
    keyword1 \sep 
    keyword2
  \end{keyword}
  
 \end{frontmatter}

\hypertarget{introduction}{%
\section{Introduction}\label{introduction}}

Spatial distribution of transport disadvantage, gaps in transit supply
and accessibility, and related issues have been addressed in much
previous research\footnote{See for example \citet{Ricciardi2015}
  \citet{Currie2003Hobart}; \citet{currie2010identifying};
  \citet{Fransen2015Identifying}; \citet{Guzman2017Assessing};
  \citet{Jaramillo2012}; \citet{Preston2007};
  \citet{Delbosc2011Transportproblems}; \citet{Delbosc2011Thespatial};
  \citet{Engels2011Social}; \citet{Pavkova2016};
  \citet{Delbosc2011Using}; \citet{Murray2001};
  \citet{Currie2010Modeling}; \citet{Currie2007Investigating};
  \citet{Currie2007Identifying}; \citet{Yigitcanlar2007};
  \citet{Wu2003}; \citet{Currie2013Exploring};
  \citet{Preston2007Accessibility}; \citet{Hurni2005};
  \citet{Mamun2011}; \citet{El_geneidy2016}; \citet{Kaplan2014};
  \citet{Martens2012}; \citet{Lucas2016AMethod};
  \citet{Liu2012Accessibility}; \citet{Lucas2012Transport_and_social};
  \citet{Lei2010Mapping}; \citet{Mavoa2012GIS};
  \citet{Delmelle2012Evaluating}; \citet{Foth2014Toward};
  \citet{Welch2013Equity}; \citet{Bell2007Travel};
  \citet{Jaramillob2011Urban}; \citet{Guzman2017Assessing};
  \citet{Wee2011Discussing}; \citet{Currie2004Gap};
  \citet{Engels2011Social}; \citet{Litman2002Evaluation};
  \citet{Parolin2017Identifying}; \citet{Xia2016AMulti};
  \citet{Welch2013AMeasure}; \citet{Jang2017Assessing}.}. Much of this
literature persents methods for calculating transport need and transit
supply, and then comparing the two across some geographic area. However,
such methodologies rarely appear to have been used again, in further
research, to study other cases or time periods. This may in part be
because applying an existing methodology to another location or time
period may not generate sufficient `new' knowledge for publication, or
be more practice that research. As well, describing a new methodology in
a research paper might only require presentation of enough results to
demonstrate the concepts, rather than widespread application to multiple
geographic contexts and development of software tools to facilitate
broader use.

An example is provided by the \citet{Currie2003Hobart}
\citet{Currie2004Gap}, \citet{Currie2007Identifying},
\citet{currie2010identifying} studies on spatial gaps between the social
need for transport and the supplied transit levels. This work presented
a transit Supply Index (SI) and compared it to measures of social need
for transport, but it is unclear whether the problems relating to social
needs and transit supply identified this previous research in the almost
two decades since the original analysis. Nor is it clear whether the
identified spatial patterns of transit need, supply and gap in these
cities are generalizable to other places as the
\citet{currie2010identifying} approach does not appear to have been
widely used since those studies. This is perhaps in part because at the
time it was first published the data needed to calculate the transit
Supply Index (SI) for a particular location was not typically readily
available. The \citet{currie2010identifying} analysis of Melbourne was
based on combining multiple operator databases and service frequency
data that had been manually extracted from transit agency websites.
Similar studies in Hobart, Perth and elsewhere in Australia \citep[
\citet{Currie2004Gap}, \citet{Currie2007Identifying},
\citet{currie2010identifying}]{Currie2003Hobart} appear likely to have
required bespoke data collection, cleaning and analysis efforts, given
that different oppoerators used different scheduling software.

Nowadays, however, the General Transit Feed Specification (GTFS) allows
timetable data to be published in a standardized format, with more than
10,000 agencies releasing data this way \citep{GTFS}. Various tools for
analysing GTFS data are now available, but there does not appear to have
been many developed to allow the analysis of spatial gaps between the
social need for transport and the amount of transit that is supplied.\\
While the previous literature provides a wealth of methodologies, the
availability of tools that might be used by researchers, practitioners
and advocates to use these approaches with GTFS data relativey easily
appears to be limited.

These gaps provide the motivation for the research reported in this
paper, in which a new R package (gtfssupplyindex) specifically developed
to calculate SI scores is presented. The paper also reports results for
Greater Melbourne in 2016 and 2021, matching the most recent censuses
and allowing comparison to the 2006 result reported in
\citet{currie2010identifying}.\\
Comparisons are also made to other parts of Australia, so as to explore
whether findings about Greater Melbourne are generalizable.

The remainder of this paper is structured as follows: the next section
outlines the background to this research, including the formulation of
the Transit Supply Index (SI), and an explanation of the GTFS. Section 3
then describes the study methodology, followed by presentation of
results in Section 4. Section 5 discusses the results and the
limitations of this study, and outlines directions for future research.
A brief conclusion is provided in Section 6.

\hypertarget{background}{%
\section{Background}\label{background}}

\hypertarget{transit-metrics}{%
\subsection{Transit metrics}\label{transit-metrics}}

Even a brief search reveals many metrics available for benchmarking
transit services. Examples include those in: the extensive Transit
Cooperative Research Program (TCRP) Report 88 guidebook on developing
performance-measurement systems, \citep{Ryus:2003aa}; and those used
across benchmarking databases and programs\\
\citep{Florida-Transit-Information-System:2018aa, UITP:2015aa, Imperial-College-London:2023aa};
and The Fielding Triangle \citep{FieldingGordonJ1987Mpts} provides a
framework for combining indicators of service inputs, outputs and
consumption to describe cost efficiency, cost effectiveness and service
effectiveness. More broadly: \citet{Litman:2003ab} and
\citet{Litman:2016aa} discuss some of the traffic, mobility,
accessibility, social equity, strategic planning and other rational
decision-making-based perspectives underling transport indicators;
\citet{Reynolds:2017ah} extends these into models of how
institutionalism, incrementalism and other public policy analysis
concepts might apply to decision-making processes relating to transit
prioritization; \citet{GuzmanLuisA.2017Aeit}, developed a measure of
accessibility in the context of policy development and social equity for
Latin American Bus Rapid Transit (BRT) networks; and
\citet{Creutzig2020streetspaceallocation} introduced street space
allocation metrics based around 10 ethical principles

However, many of these, and other, transit metrics may be difficult to
calculate, and/or complex to explain or understand, especially for those
who are not planners, engineers or other technical specialists. Where
pre-calculated metrics are immediately available it may not be possible
for practitioners, researchers or advocates to independently generate
scores so as to test proposed system changes, or demonstrate impacts to
politicans, the general public or others.

Contrasting examples are provided by the metrics in the Transit Capacity
and Quality of Service Manual (TCQSM) and the Transit Score metric,
readily available on the \citet{WalkScore:2023tg} website, A Transit
Score is available for locations with a published GTFS feed, eliminating
the need for any calculations. The meaning of these Transit Scores also
appears easy to explain, with the highest possible score of 100
representing what might be experienced in the center of New York.
However, the Transit Score algorithm is unpublished, and effectively a
black box. It does not appear that Transit Scores can be calculated
independently or generated for proposed changes to networks. In
contrast, the TCQSM provides a wide range of metrics for measuring
different aspects of a transit system. The TCQSM scores themselves
appear easy to understand or explain, ranging from A to F, although the
number of metrics is very large and this might limit the practicality of
using the TCQSM in practice for communicating with non-technical
audiences. All of these can be calculated independently, given
sufficient data, and \citet{Wong:2013aa} provides an example reporting
various TCQSM metrics across 50 transit operators. This analysis by
\citet{Wong:2013aa} is made possible by the availability of General
Transit Feed Specification (GTFS) datasets for each of the transit
systems

The GTFS is an open, text-based format, developed originally to allow
transit to be included in the Google Maps navigation platform
\citep{GTFS}. Figure @ref(fig:GTFS\_ERD) shows an Entity Relationship
Diagram (ERD) of the GTFS data structure, indicatating how GTFS data is
stored as a series of tables (agency, routes, trips etc.) with primary
and foreign keys (agency\_id, route\_id, trip\_id etc.) providing links.
While there are many software tools for analyzing, visualizing or
otherwise manipulating GTFS data, one to calculate Transit Supply Index
(SI) scores is not yet available.

\begin{figure}
\includegraphics[width=1\linewidth]{graphics/GTFS} \caption{GTFS entity relationship diagram. Source: adapted by author from Alamri et al (2023) and the GTFS Schedule Reference (16/11/2023 revision).}\label{fig:GTFS_ERD}
\end{figure}

\hypertarget{the-transit-suppy-index}{%
\subsection{The Transit Suppy Index}\label{the-transit-suppy-index}}

A generalized form of the SI equation, adapted from
\citet{currie2010identifying}, is:

\[SI_{area, time} = \sum{\frac{Area_{Bn}}{Area_{area}}*SL_{n, time}}\]

where:

\begin{itemize}
\item
  \(SI_{area, time}\) is the Supply Index for the area of interest and a
  given period of time;
\item
  \(Area_{Bn}\) is the buffer area for each stop (n) within the area of
  interest (in \citet{currie2010identifying} this was based on a radius
  of 400 metres for bus and tram stops, and 800 metres for railway
  stations);
\item
  \(Area_{area}\) is the area of the area of interest; and
\item
  \(SL_{n,time}\) is the number of transit arrivals for each stop for a
  given time period.
\end{itemize}

\begin{figure}
\includegraphics[width=1\linewidth]{graphics/Currie2010SI} \caption{Distribution of supply measure scores – Metropolitan Melbourne (2006), Source: Currie (2010)}\label{fig:Currie_map_SI}
\end{figure}

Figure \ref{fig:Currie_map_SI} shows a map of SI scores across Greater
Melbourne in 2006, which was included in \citet{currie2010identifying}.
The general patterns appear to be higher levels of transit supply closer
to the city's centre and along passenger railway lines, and outer areas
with very low SI scores or no transit supply at all.

\hypertarget{social-need-and-needs-gap}{%
\subsection{Social need and needs gap}\label{social-need-and-needs-gap}}

\citet{currie2010identifying} also assessed the social need for transit
across Greater Melbourne using: the Australian Bureaus of Statistics'
Index of Related Socio-Economic Advantage/Disadvantage (IRSAD); a
transport needs index derived by \citet{currie2010identifying} from
eight weighted indicators; and a combination of the two.

Figures \ref{fig:Currie_chart_gap} and \ref{fig:Currie_map_gap}
reproduce a chart comparing transport needs and transit supply, and a
map of areas with very high social needs but zero or very low transit
supply.

\begin{figure}
\includegraphics[width=1\linewidth]{graphics/Currie2010chart} \caption{Log supply score and need index values – Melbourne needs-gap study, Source: Currie (2010)}\label{fig:Currie_chart_gap}
\end{figure}

\begin{figure}
\includegraphics[width=1\linewidth]{graphics/Currie2010gap} \caption{Melbourne needs-gap – very high transport need areas with zero or very low public transport supply, Source: Currie (2010)}\label{fig:Currie_map_gap}
\end{figure}

The results indicated service gaps of concern, especially in outer parts
of Melbourne where low density development patterns make provision of
transit more challenging. \citet{currie2010identifying} found that
``(o)verall, 8.2\% of Melbourne residents have `very high' needs but
`zero', `low' or `very low' public transport supply.'', and suggested
that this approach was ``substantially more useful than the presentation
of anecdotal evidence, which is the most common means of identifying
transport needs in local transport studies throughout the world.''

However, it doesn't appear that this approach has been widely adopted in
practice or academia.\\
Our suspicion is that while the SI has a relatively simple formula and
requires only geographic and timetable data, the lack of a software tool
to perform these calculations may be part of the reason that it has not
been more widely adopted and why formal needs-supply-gap analysis may
still be uncommon.

It is also unclear whether the patterns in Melbourne, where areas with
very high transport needs but zero or very low transit supply tend to be
in outer areas serviced by buses, are similar to patterns in other
cities. Nor is it clear whether the patterns in Melbourne itself have
changed since the 2006 analysis.

Developing a software tool, and then using it to comparing current
conditions and other locations to the findings of
\citet{currie2010identifying}, therefore, is the primary aim of this
paper.

\hypertarget{methodology}{%
\section{Methodology}\label{methodology}}

This study developed a R programming language \citep{R-base} package of
tools for calculating the SI from GTFS data \citet{wickham2023r}
informed the package setup and development approach. Various existing
packages were relied upon including: the sf package \citep{R-sf} for
geospatial analysis; the tidyverse \citep{tidyverse2019}; gtfstools
\citep{R-gtfstools}; and tidytransit \citep{R-tidytransit}. Some code
was adapted from examples, vignettes and other documentation in these
and other packages.

Two cases were used during the code development and testing, such that
results might be generated for real GTFS data: the Mornington Peninsula
Tourist Railway GTFS feed; and the Public Transport Victoria (PTV) GTFS
feed, both in Victoria, Australia. Both were selected primarily for
convenience, given that the authors are familiar with the typical
service patterns and geography. Adopting the Mornington Peninsula
Tourist Railway network, which consists of only three stations, also
facilitated hand calculation of the SI as a cross-check of the results
produced by the developed package.

Figure @ref(Melbourne\_map)) shows the areas of interest relevant to the
code development and testing, and selected railway stations. Statistical
Area (SA) zones were adopted from the Australian Bureau of Statistics
\citep{ABSmaps} as the areas of interest, and included SA3
zones\footnote{These are generally similar to Local Government Area
  (LGA) boundaries.} across the Greater Melbourne Greater Capital City
statistical area (main); and SA1 zones\footnote{SA1 zones are the
  smallest geographical areas for which results are reported in the
  Australian census.} within 800 metres of the Mornington Penninsula
railway (right).

\begin{figure}
\includegraphics[width=1\linewidth]{graphics/all_maps} \caption{Areas of interest}\label{fig:Melbourne_map}
\end{figure}

\hypertarget{mornington-penninsula-tourist-railway}{%
\subsection{Mornington Penninsula Tourist
Railway}\label{mornington-penninsula-tourist-railway}}

The Morning Peninsula Tourist Railway is in the outer south-east of
Melbourne, running on Sundays and Wednesdays between Mornington and
Moorooduc, with an intermediate stop at Tanti Park (see
https://transitfeeds.com/p/mornington-railway/806/latest/stops). A GTFS
feed from 2018 was selected for the purposes of tests and demonstrating
the code and output. Australian Bureau of Statistics (ABS) data was also
used, sources via the strayr and absmapsdata packages \citep{r-strayr}.
The Mornington Peninsular Statistical Area 3 (SA3) zone and the
Statistical Area 1 (SA1) zones contained within it were adopted as the
areas of interest.

\hypertarget{public-transport-victoria-ptv}{%
\subsection{Public Transport Victoria
(PTV)}\label{public-transport-victoria-ptv}}

The Victorian GTFS feed, published by Public Transport Victoria (PTV)
and with historical feeds sourced via
\citet{transitfeeds_victoria:2023aa}, was used for analysis of Victoria.
SI scores were obtained for the weeks starting on the day of the census
in 2016 and 2021, which were on Tuesday 9th and 10th of August
respectively.

\hypertarget{social-disadvantage-measurement-approach}{%
\subsection{Social disadvantage measurement
approach}\label{social-disadvantage-measurement-approach}}

This paper adopts the same approach to social disadvantage measurement
as in \citet{currie2010identifying}.

\begin{quote}
\begin{quote}
\begin{quote}
\begin{quote}
\begin{quote}
\begin{quote}
Continue from
here\textgreater\textgreater\textgreater\textgreater\textgreater\textgreater{}
\end{quote}
\end{quote}
\end{quote}
\end{quote}
\end{quote}
\end{quote}

\hypertarget{results}{%
\section{Results}\label{results}}

\hypertarget{code-structure-and-functionality}{%
\subsection{Code structure and
functionality}\label{code-structure-and-functionality}}

Developed code is available and documented on github
\citep{gtfssupplyindex_github}. The structure of the package, functions
developed, and data tables are shown in Figure @ref(fig:SI\_ERD). This
shows\\
how the package takes input from three files: a gtfs feed (gtfs.zip); a
sf object describing the geometry of the areas for which the SI is to be
calculated; and a csv file (included in the package) defining the buffer
zone distances for each route type. The ultimate output is a
si\_by\_area\_and\_hour table (bottom-right), which reports the SI score
for each hour of the day across dates specified by the user.

\begin{figure}
\includegraphics[width=1\linewidth]{graphics/SI_data_structure} \caption{Entity Relationship Diagram (ERD) showing the data structure and functions related to the gtfssupplyindex package}\label{fig:SI_ERD}
\end{figure}

Various functions and their output are explained in the following, using
the Mornington Peninsula GTFS for December 30th, 2018, and SA1 zone
boundaries as a worked example. Individual steps are:

\begin{enumerate}
\def\labelenumi{(\arabic{enumi})}
\item
  loading the gtfs.zip file: the gtfs\_by\_route\_type function loads
  the gtfs data and splits it into a list (by route\_type) of tidygtfs
  objects, using the filter\_by\_route\_type function from the gtfstools
  package \citep{filter_GTFS_by_mode}.
\item
  loading geometry information about the areas of interest: geographical
  data about the areas of interest are loaded by the
  load\_areas\_of\_interest.R function into an sf object, using the sf
  package \citep{R-sf}. The resultant areas\_of\_interest table contains
  each area\_id and its associated geometry. Data about buffer zones,
  specifically the walking distance threshold assigned to each
  route\_type (mode) is then loaded, again through a function
  (load\_buffer\_zone.R).
\item
  calculating which stops are within the catchment walking distance of
  which areas: using the stops\_in\_walk\_dist function. Figure
  @ref(fig:calculate\_stop\_in\_or\_near\_areas\_verbose)) shows how
  this function identified SA1 areas within the 800 metre catchment of
  the three Mornington stations.
\end{enumerate}

\begin{figure}
\includegraphics[width=1\linewidth]{Leveraging_GTFS_to_assess_transit_supply_Transport_Geography_files/figure-latex/calculate_stop_in_or_near_areas_verbose-1} \caption{Step 3, stop catchments for the Mornington Penninsula Tourist Railway, showing intersections with SA1 zones}\label{fig:calculate_stop_in_or_near_areas_verbose}
\end{figure}

\begin{enumerate}
\def\labelenumi{(\arabic{enumi})}
\setcounter{enumi}{3}
\tightlist
\item
  Calculating SI scores for a given time period: The si\_calc.R function
  calculates the number of arrivals in a given time period, using code
  adapted from an article included in the tidytransit package
  \citep{tidytransit_departure_timetable}, and combines this with the
  calculated area components. The si\_total.R and hourly.R functions
  provided aggregation, giving the results mapped in Figure
  @ref(fig:SI\_mornington\_20181230\_output).
\end{enumerate}

\begin{figure}
\centering
\includegraphics{Leveraging_GTFS_to_assess_transit_supply_Transport_Geography_files/figure-latex/SI_mornington_20181230_output-1.pdf}
\caption{Mornington Penninsula Tourist Railway hourly SI values for
December 30, 2018}
\end{figure}

\hypertarget{si-scores}{%
\subsection{SI scores}\label{si-scores}}

\begin{figure}
\centering
\includegraphics{Leveraging_GTFS_to_assess_transit_supply_Transport_Geography_files/figure-latex/Greater_Melbourne_2016_2021-1.pdf}
\caption{SI scores, census day 2016 and 2021}
\end{figure}

\hypertarget{imrad}{%
\subsubsection{IMRAD}\label{imrad}}

\hypertarget{comparing-cases}{%
\subsection{Comparing cases}\label{comparing-cases}}

\hypertarget{population-and-equality}{%
\subsubsection{Population and equality}\label{population-and-equality}}

\hypertarget{purpose-of-transit-in-the-citys-transport-policy}{%
\subsection{Purpose of transit in the city's transport
policy}\label{purpose-of-transit-in-the-citys-transport-policy}}

\hypertarget{indexes-and-comparing-cities}{%
\subsection{Indexes and comparing
cities}\label{indexes-and-comparing-cities}}

\hypertarget{discussion}{%
\section{Discussion}\label{discussion}}

\hypertarget{limitations}{%
\subsection{Limitations}\label{limitations}}

\hypertarget{directions-for-furture-research}{%
\subsection{Directions for furture
research}\label{directions-for-furture-research}}

\hypertarget{conclusions}{%
\section{Conclusions}\label{conclusions}}

\renewcommand\refname{References}
\bibliography{References.bib, packages.bib}


\end{document}
