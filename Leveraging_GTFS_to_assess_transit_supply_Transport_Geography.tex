\documentclass[preprint, 3p,
authoryear]{elsarticle} %review=doublespace preprint=single 5p=2 column
%%% Begin My package additions %%%%%%%%%%%%%%%%%%%

\usepackage[hyphens]{url}

  \journal{Transport Geography?} % Sets Journal name

\usepackage{graphicx}
%%%%%%%%%%%%%%%% end my additions to header

\usepackage[T1]{fontenc}
\usepackage{lmodern}
\usepackage{amssymb,amsmath}
% TODO: Currently lineno needs to be loaded after amsmath because of conflict
% https://github.com/latex-lineno/lineno/issues/5
\usepackage{lineno} % add
\usepackage{ifxetex,ifluatex}
\usepackage{fixltx2e} % provides \textsubscript
% use upquote if available, for straight quotes in verbatim environments
\IfFileExists{upquote.sty}{\usepackage{upquote}}{}
\ifnum 0\ifxetex 1\fi\ifluatex 1\fi=0 % if pdftex
  \usepackage[utf8]{inputenc}
\else % if luatex or xelatex
  \usepackage{fontspec}
  \ifxetex
    \usepackage{xltxtra,xunicode}
  \fi
  \defaultfontfeatures{Mapping=tex-text,Scale=MatchLowercase}
  \newcommand{\euro}{€}
\fi
% use microtype if available
\IfFileExists{microtype.sty}{\usepackage{microtype}}{}
\usepackage[]{natbib}
\bibliographystyle{plainnat}

\usepackage{graphicx}
\ifxetex
  \usepackage[setpagesize=false, % page size defined by xetex
              unicode=false, % unicode breaks when used with xetex
              xetex]{hyperref}
\else
  \usepackage[unicode=true]{hyperref}
\fi
\hypersetup{breaklinks=true,
            bookmarks=true,
            pdfauthor={},
            pdftitle={Leveraging GTFS to explore spatial patterns in transit supply with respect to social needs},
            colorlinks=false,
            urlcolor=blue,
            linkcolor=magenta,
            pdfborder={0 0 0}}

\setcounter{secnumdepth}{5}
% Pandoc toggle for numbering sections (defaults to be off)


% tightlist command for lists without linebreak
\providecommand{\tightlist}{%
  \setlength{\itemsep}{0pt}\setlength{\parskip}{0pt}}




\usepackage{subfig}
\usepackage{booktabs}
\usepackage{longtable}
\usepackage{array}
\usepackage{multirow}
\usepackage{wrapfig}
\usepackage{float}
\usepackage{colortbl}
\usepackage{pdflscape}
\usepackage{tabu}
\usepackage{threeparttable}
\usepackage{threeparttablex}
\usepackage[normalem]{ulem}
\usepackage{makecell}
\usepackage{xcolor}



\begin{document}


\begin{frontmatter}

  \title{Leveraging GTFS to explore spatial patterns in transit supply
with respect to social needs}
    \author[Public Transport Research Group (PTRG)]{James Reynolds%
  %
  \fnref{1}}
   \ead{james.reynolds@monash.edu} 
    \author[Public Transport Research Group (PTRG)]{Yanda Qu%
  %
  \fnref{2}}
   \ead{yanda.qu@monash.edu} 
    \author[Public Transport Research Group (PTRG)]{Graham Currie%
  \corref{cor1}%
  \fnref{3}}
   \ead{graham.currie@monash.edu} 
      \affiliation[Public Transport Research Group (PTRG)]{
    organization={Public Transport Research Group (PTRG), Institute of
Transport Studies, Department of Civil Engineering Engineering, Monash
University},addressline={Clayton
Campus},city={Melbourne},postcode={3800},state={Victoria},country={Australia},}
    \cortext[cor1]{Corresponding author}
    \fntext[1]{Research Fellow}
    \fntext[2]{PhD Student}
    \fntext[3]{Professor}
  
  \begin{abstract}
  This is the abstract.

  It consists of two paragraphs.
  \end{abstract}
    \begin{keyword}
    keyword1 \sep 
    keyword2
  \end{keyword}
  
 \end{frontmatter}

\section{Introduction}\label{introduction}

\citet{Currie2003Hobart}, \citet{Currie2004Gap},
\citet{Currie2007Identifying} and \citet{currie2010identifying} explored
spatial gaps between the social need for transport and what transit is
supplied. This work presented a transit Supply Index (SI) and compared
it to measures of social need for transport across a few Australian
cities.

However, in almost two decades since the SI was developed there does not
appear to have been much further use of this approach. It is unclear
whether gaps between social needs and transit supply have increased or
reduced. Nor is it clear whether the spatial patterns identified in
Hobart \citep{Currie2003Hobart, Currie2004Gap} and Melbourne
\citep{Currie2007Identifying, currie2010identifying}, are similar in
other places

This may in part be because applying an existing methodology to another
location or time period might not generate sufficient `new' knowledge
for publication, and/or be more an activity for practitioners than for
researchers. As well, at the time applying such methodologies elsewhere
would appear likely to have required a bespoke data collection, cleaning
and analysis effort. Nowadays, however, the General Transit Feed
Specification (GTFS) allows timetable data to be published in a
standardized format. More than 10,000 transit agencies release data this
way \citep{GTFS}. Various tools for analysing GTFS data are now
available, but there does not appear to have been many developed to
allow the analysis of spatial gaps between the social need for transport
and the amount of transit that is supplied. While the previous
literature provides a wealth of methodologies, the availability of tools
that might be used by researchers, practitioners and advocates to use
these approaches with GTFS data relatively easily appears limited.

This gap provides the motivation for the research reported in this
paper, in which a new R package (gtfssupplyindex) specifically developed
to calculate SI scores is presented. The paper also reports results for
Greater Melbourne in 2016 and 2021, matching the most recent censuses
and allowing comparison to the 2006 result reported in
\citet{currie2010identifying}. Comparisons are also made to other parts
of Australia, so as to explore whether findings about Greater Melbourne
are generalizable.

The remainder of this paper is structured as follows: the next section
outlines the background to this research, including the formulation of
the Transit Supply Index (SI), and an explanation of the GTFS. Section 3
then describes the study methodology, followed by presentation of
results in Section 4. Section 5 discusses the results and the
limitations of this study, and outlines directions for future research.
A brief conclusion is provided in Section 6.

\section{Background}\label{background}

\subsection{Transit metrics}\label{transit-metrics}

Even a brief search reveals many metrics available for benchmarking
transit services. Examples include those in: the extensive Transit
Cooperative Research Program (TCRP) Report 88 guidebook on developing
performance-measurement systems \citep{Ryus:2003aa}; and those used
across benchmarking databases and programs
\citep{Florida-Transit-Information-System:2018aa, UITP:2015aa, Imperial-College-London:2023aa}.
The Fielding Triangle \citep{FieldingGordonJ1987Mpts} provides a
framework for combining indicators of service inputs, outputs and
consumption to describe cost efficiency, cost effectiveness and service
effectiveness. More broadly: \citet{Litman:2003ab} and
\citet{Litman:2016aa} discuss some of the traffic, mobility,
accessibility, social equity, strategic planning and other rational
decision-making-based perspectives underling transport indicators;
\citet{Reynolds:2017ah} extends these into models of how
institutionalism, incrementalism and other public policy analysis
concepts might apply to decision-making processes relating to transit
prioritization; \citet{GuzmanLuisA.2017Aeit}, developed a measure of
accessibility in the context of policy development and social equity for
Latin American Bus Rapid Transit (BRT) networks; and
\citet{Creutzig2020streetspaceallocation} introduced street space
allocation metrics based around 10 ethical principles.

However, many of these, and other, transit metrics may be difficult to
calculate, and/or complex to explain or understand, especially for those
who are not planners, engineers or other technical specialists. Where
pre-calculated metrics are immediately available it may not be possible
for practitioners, researchers or advocates to independently generate
scores so as to test proposed system changes, or demonstrate impacts to
politicans, the general public or others. Contrasting examples are
provided by the metrics in the Transit Capacity and Quality of Service
Manual (TCQSM) and the Transit Score metric \citep{WalkScore:2023tg}.
Transit Scores are readily available on the Transit Score website for
locations with a published GTFS feed. The meaning of these Transit
Scores also appears easy to explain, with the highest possible score of
100 representing the sort of transit accessibility that might be
xperienced in the center of New York. However, the Transit Score
algorithm is a black box, and cannot be calculated independently or
generated for proposed changes to networks. In contrast, the TCQSM
provides a wide range of metrics for measuring different aspects of a
transit system. The TCQSM scores themselves appear easy to understand or
explain, ranging from A (good) to F (bad), and these can be calculated
independently, given sufficient data. \citet{Wong:2013aa} provides an
example of what can be done combining GTFS data with metrics that can be
independently calculating, reporting TCQSM scores across 50 transit
operators.

The GTFS is an open, text-based format, developed originally to allow
transit to be included in the Google Maps navigation platform
\citep{GTFS}. Figure @ref(fig:GTFS\_ERD) shows an Entity Relationship
Diagram (ERD) of the GTFS structure, indicatating how data is stored as
a series of tables (agency, routes, trips etc.) linked by primary and
foreign keys (agency\_id, route\_id, trip\_id etc.). While there are
many software tools for analyzing, visualizing or otherwise manipulating
GTFS data, one to calculate Transit Supply Index (SI) scores is not yet
available.

\begin{figure}
\includegraphics[width=1\linewidth]{graphics/GTFS} \caption{GTFS entity relationship diagram. Source: adapted by author from Alamri et al (2023) and the GTFS Schedule Reference (16/11/2023 revision).}\label{fig:GTFS_ERD}
\end{figure}

\subsection{The Transit Suppy Index}\label{the-transit-suppy-index}

A generalized form of the SI equation, adapted from
\citet{currie2010identifying}, is:

\[SI_{area, time} = \sum{\frac{Area_{Bn}}{Area_{area}}*SL_{n, time}}\]

where:

\begin{itemize}
\item
  \(SI_{area, time}\) is the Supply Index for the area of interest and a
  given period of time;
\item
  \(Area_{Bn}\) is the buffer area for each stop (n) within the area of
  interest (in \citet{currie2010identifying} this was based on a radius
  of 400 metres for bus and tram stops, and 800 metres for railway
  stations);
\item
  \(Area_{area}\) is the area of the area of interest; and
\item
  \(SL_{n,time}\) is the number of transit arrivals for each stop for a
  given time period.
\end{itemize}

\begin{figure}
\includegraphics[width=1\linewidth]{graphics/Currie2010SI} \caption{Distribution of supply measure scores – Metropolitan Melbourne (2006), Source: Currie (2010)}\label{fig:Currie_map_SI}
\end{figure}

\citet{currie2010identifying} reported SI scores across Greater
Melbourne in 2006, as shown in Figure \ref{fig:Currie_map_SI}. The
general patterns appear to be higher levels of transit supply in the
middle and inner suburbs and along passenger railway lines. Outer areas
tend to have very low SI scores or no transit supply at all.

\subsection{Social need and needs gap}\label{social-need-and-needs-gap}

As well as measuring transit supply, \citet{currie2010identifying} also
assessed the social need for transit across Greater Melbourne using: the
Australian Bureaus of Statistics' Index of Related Socio-Economic
Advantage/Disadvantage (IRSAD); a transport needs index derived by
\citet{currie2010identifying} from eight weighted indicators; and a
combination of the two. Figure \ref{fig:Currie_map_gap} reproduces the
resultant map identifying areas with very high transport needs, but very
low or no transit supply.

\begin{figure}
\includegraphics[width=1\linewidth]{graphics/Currie2010gap} \caption{Melbourne needs-gap – very high transport need areas with zero or very low public transport supply, Source: Currie (2010)}\label{fig:Currie_map_gap}
\end{figure}

The results indicated various areas where service gaps might be of
concern, especially in outer parts of Melbourne where low density
development patterns make provision of transit more challenging.
\citet{currie2010identifying} found that ``(o)verall, 8.2\% of Melbourne
residents have `very high' needs but `zero', `low' or `very low' public
transport supply.''\\
They also suggested that planning for transit service provision using
this approach would be ``substantially more useful than the presentation
of anecdotal evidence, which is the most common means of identifying
transport needs in local transport studies throughout the world.''
However, it doesn't appear that this approach has been widely adopted in
practice or academia. Our suspicion is that while the SI has a
relatively simple formula and requires only geographic and timetable
data, the lack of a software tool to perform these calculations may be
part of the reason that it has not been more widely adopted and why
formal needs-supply-gap analysis may still be uncommon.

It is also unclear whether the patterns in Melbourne identified in
\citet{currie2010identifying}, where areas with very high transport
needs but zero or very low transit supply tend to be in outer areas
serviced by buses, are similar to patterns in other cities. Nor is it
clear whether the patterns in Melbourne itself have changed since the
2006 analysis. Developing a software tool to calculate SI tools from
GTFS data, and then using it to comparing current conditions and other
locations to the findings of \citet{currie2010identifying}, therefore,
is the primary aim of this paper.

\section{Methodology}\label{methodology}

This study developed a package of tools for calculating the SI from GTFS
data using the R programming language \citep{R-base}. The
recommendations of \citet{wickham2023r} informed the package setup and
development approach. Various existing packages were relied upon
including: the sf package \citep{R-sf} for geospatial analysis; the
tidyverse \citep{tidyverse2019}; gtfstools \citep{R-gtfstools}; and
tidytransit \citep{R-tidytransit}. Australian Bureau of Statistics (ABS)
data was also used, sourced via the strayr and absmapsdata packages
\citep{r-strayr}. Some code was adapted from examples, vignettes and
other documentation in these and other packages.

Two cases were used during the code development and testing, such that
results might be generated for real GTFS data: the Mornington Peninsula
Tourist Railway GTFS feed; and the Public Transport Victoria (PTV) GTFS
feed, both in Victoria, Australia. Both were selected primarily for
convenience, given that the authors are familiar with the typical
service patterns and geography. Adopting the Mornington Peninsula
Tourist Railway network, which consists of only three stations, also
facilitated hand calculation of the SI as a cross-check of the results
produced by the developed package.

Figure @ref(Melbourne\_map)) shows the areas of interest relevant to the
code development and testing, and selected railway stations. Statistical
Area (SA) zones were adopted from the Australian Bureau of Statistics
\citep{ABSmaps} as the areas of interest, and included SA3
zones\footnote{These are generally similar to Local Government Area
  (LGA) boundaries.} across the Greater Melbourne Greater Capital City
statistical area (Figure @ref(Melbourne\_map), left); and SA1
zones\footnote{SA1 zones are the smallest geographical areas for which
  results are reported in the Australian census.} within 800 metres of
the Mornington Penninsula railway (Figure @ref(Melbourne\_map), right).

\begin{figure}
\includegraphics[width=1\linewidth]{graphics/all_maps} \caption{Areas of interest}\label{fig:Melbourne_map}
\end{figure}

\subsection{Mornington Penninsula Tourist
Railway}\label{mornington-penninsula-tourist-railway}

The Morning Peninsula Tourist Railway is in the outer south-east of
Melbourne, running on Sundays and Wednesdays between Mornington and
Moorooduc, with an intermediate stop at Tanti Park\footnote{see
  https://transitfeeds.com/p/mornington-railway/806/latest/stops}. A
GTFS feed from 2018 was selected for the purposes of tests, and for
demonstrating the code and outputs reported here.

\subsection{Public Transport Victoria
(PTV)}\label{public-transport-victoria-ptv}

The Victorian GTFS feed, published by Public Transport Victoria (PTV)
and with historical feeds sourced via
\citet{transitfeeds_victoria:2023aa}, was used for analysis of Greater
Melbourne and Victoria. SI scores were obtained for the weeks starting
on the day of the census in 2016 and 2021, which were on Tuesday 9th and
10th of August respectively.

\subsection{Australian Capital
Territory}\label{australian-capital-territory}

\subsection{Adelaide}\label{adelaide}

\subsection{Metro Tasmania Burnie}\label{metro-tasmania-burnie}

\subsection{Metro Tasmania Hobart}\label{metro-tasmania-hobart}

\subsection{Metro Tasmania Launceston}\label{metro-tasmania-launceston}

\subsection{Translink South East
Queensland}\label{translink-south-east-queensland}

\subsection{Transperth}\label{transperth}

\section{Results}\label{results}

\subsection{Code structure and
functionality}\label{code-structure-and-functionality}

Developed code is available and documented on github (see
\citet{gtfssupplyindex_github}). The structure of the package, functions
developed, and data tables are shown in Figure @ref(fig:SI\_ERD), which
indicates how the package takes input from three files: a gtfs feed
(gtfs.zip); a sf object describing the geometry of the areas of interest
for which the SI is to be calculated; and a csv file (included in the
package) defining the buffer zone distances for each route type. The
ultimate output is a si\_by\_area\_and\_hour table (Figure
@ref(fig:SI\_ERD), bottom-right), which reports the SI score for each
hour of a day specified by the user.

\begin{figure}
\includegraphics[width=1\linewidth]{graphics/SI_data_structure} \caption{Entity Relationship Diagram (ERD) showing the data structure and functions related to the gtfssupplyindex package}\label{fig:SI_ERD}
\end{figure}

The various functions included in the package and their output are
explained in the following, using the Mornington Peninsula GTFS as a
worked example\footnote{This paper itself was prepared in Rmarkdown and
  is available at
  https://github.com/James-Reynolds/gtfssupplyindex\_main\_paper. Code
  snippets used to produce the outputs of the worked example can be
  viewed there.} Individual steps are as follows:

\begin{enumerate}
\def\labelenumi{(\arabic{enumi})}
\item
  The GTFS data is loaded using\\
  the gtfs\_by\_route\_type function which also splits it into a list
  (by route\_type) of tidygtfs objects, using the
  filter\_by\_route\_type function from the gtfstools package
  \citep{filter_GTFS_by_mode}.
\item
  Geometry information about the areas of interest is loaded by the
  load\_areas\_of\_interest function using the sf package \citep{R-sf}.
  The resultant areas\_of\_interest table
\item
  The walking distance threshold assigned to each route\_type (mode) is
  then loaded, using the load\_buffer\_zone function.
\item
  Stops within the catchment walking distance of each area\_of\_interest
  are then identified using the stops\_in\_walk\_dist function. Figure
  @ref(fig:calculate\_stop\_in\_or\_near\_areas\_verbose) shows how this
  functions identifies the SA1 areas within the 800 metre catchment of
  all three of the Mornington stations.
\end{enumerate}

\begin{figure}
\includegraphics[width=1\linewidth]{Leveraging_GTFS_to_assess_transit_supply_Transport_Geography_files/figure-latex/calculate_stop_in_or_near_areas_verbose-1} \caption{Step 3, stop catchments for the Mornington Penninsula Tourist Railway, showing intersections with SA1 zones}\label{fig:calculate_stop_in_or_near_areas_verbose}
\end{figure}

\begin{enumerate}
\def\labelenumi{(\arabic{enumi})}
\setcounter{enumi}{4}
\tightlist
\item
  SI scores are calculated for a given time period using the si\_calc
  function, which first identifies the number of arrivals in a given
  time period at each stop using code adapted from an article included
  in the tidytransit package \citep{tidytransit_departure_timetable}.
  The area terms of the SI value are calculated, with the si\_total and
  hourly functions providing aggregation by area or by area and hour.
  Figure @ref(fig:SI\_mornington\_20181230\_output) shows the hourly
  output for the Mornington Peninsula Railway, adopting December 30,
  2018 as the date of analysis.
\end{enumerate}

\begin{figure}
\centering
\includegraphics{Leveraging_GTFS_to_assess_transit_supply_Transport_Geography_files/figure-latex/SI_mornington_20181230_output-1.pdf}
\caption{Mornington Penninsula Tourist Railway hourly SI values for
December 30, 2018}
\end{figure}

\subsection{SI scores}\label{si-scores}

\subsubsection{Greater Melbourne}\label{greater-melbourne}

\begin{figure}
\centering
\includegraphics{Leveraging_GTFS_to_assess_transit_supply_Transport_Geography_files/figure-latex/Greater_Melbourne_2016_2021-1.pdf}
\caption{SI scores, census day 2021}
\end{figure}

Figure \ref{fig:Greater_Melbourne_2016_2021} shows SI values on census
day for Greater Melbourne in 2016 and 2021. Comparison to Figure
\ref{fig:Currie_map_SI} indicates that:

\begin{itemize}
\tightlist
\item
  The Greater Melbourne Greater Capital City Statistical Area now covers
  a larger area than it did in 2006, with the northern boundary having
  shifted northwards by up to around 40 kilometres \footnote{See
    https://maps.abs.gov.au/?xmin=15884115.813179802\&ymin=-4689558.173698483\&xmax=16551869.692279067\&ymax=-4397262.977535984\&bottomlayer=ASGS2021:GCCSA\&toplayer=ASGC2006:SD}.
\item
  There are still many areas, typically in the outer parts of Melbourne,
  where there is very low or zero transit supply.
\item
  A similar pattern of higher transit supply near railway lines is
  evident, but there appears to be more areas with above average or
  better service levels further out from the centre of the city.
\item
  In general, above average, high and very high service levels appear to
  be more spread out, especially into the middle suburbs.
\end{itemize}

Comparing 2016 and 2021 suggests that the geographic spread of service
levels has remained largely the same, albeit with some areas that had
little to no transit service appearing to have new services.

\begin{tabular}{l|l|l}
\hline
**Characteristic** & **2016**, N = 10,723 & **2021**, N = 10,999\\
\hline
SI & NA & NA\\
\hline
Mean (SD) & 489 (1,019) & 486 (982)\\
\hline
Median (25\%,75\%) & 229 (92,524) & 237 (95,523)\\
\hline
Range & 0, 15,781 & 0, 15,346\\
\hline
\end{tabular}

\begin{figure}
\centering
\includegraphics{Leveraging_GTFS_to_assess_transit_supply_Transport_Geography_files/figure-latex/Greater_Melbourne_2016_2021_year_SI-1.pdf}
\caption{SI scores by SA3, census day 2016 and 2021}
\end{figure}

Figure \ref{fig:Greater_Melbourne_2016_2021_year_SI} compares SI scores
for Greater Melbourne across 2016 and 2021. Results indicate a
statistically significant difference, but there does not appear to have
been much change across median, mean, interquartile range or overall
range (See table)

\begin{figure}
\centering
\includegraphics{Leveraging_GTFS_to_assess_transit_supply_Transport_Geography_files/figure-latex/Greater_Melbourne_2016_2021_scatterplot-1.pdf}
\caption{SI scores by SA3, census day 2016 and 2021}
\end{figure}

Figure \ref{fig:Greater_Melbourne_2016_2021_scatterplot} compares the
2016 and 2021 SI scores for SA1 zones across Greater Melbourne. Results
indicate a statistically significant relationship, and for most SA1
zones it appears that SI scores have remained largely the same in 2021
as they were in 2016. Of interest, however, may be a group of SA1s that
had SI scores in the range of 3-4,000 in 2016, but only around 2,500 in
2021.

\subsubsection{IMRAD}\label{imrad}

\subsection{Comparing cases}\label{comparing-cases}

\subsubsection{Population and equality}\label{population-and-equality}

\subsection{Purpose of transit in the city's transport
policy}\label{purpose-of-transit-in-the-citys-transport-policy}

\subsection{Indexes and comparing
cities}\label{indexes-and-comparing-cities}

\section{Discussion}\label{discussion}

\subsection{Limitations}\label{limitations}

\subsection{Directions for furture
research}\label{directions-for-furture-research}

\section{Conclusions}\label{conclusions}

\renewcommand\refname{References}
\bibliography{References.bib, packages.bib}


\end{document}
