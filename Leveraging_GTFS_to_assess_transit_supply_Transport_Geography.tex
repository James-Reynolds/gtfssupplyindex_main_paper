\documentclass[preprint, 3p,
authoryear]{elsarticle} %review=doublespace preprint=single 5p=2 column
%%% Begin My package additions %%%%%%%%%%%%%%%%%%%

\usepackage[hyphens]{url}

  \journal{Transport Geography?} % Sets Journal name

\usepackage{graphicx}
%%%%%%%%%%%%%%%% end my additions to header

\usepackage[T1]{fontenc}
\usepackage{lmodern}
\usepackage{amssymb,amsmath}
% TODO: Currently lineno needs to be loaded after amsmath because of conflict
% https://github.com/latex-lineno/lineno/issues/5
\usepackage{lineno} % add
\usepackage{ifxetex,ifluatex}
\usepackage{fixltx2e} % provides \textsubscript
% use upquote if available, for straight quotes in verbatim environments
\IfFileExists{upquote.sty}{\usepackage{upquote}}{}
\ifnum 0\ifxetex 1\fi\ifluatex 1\fi=0 % if pdftex
  \usepackage[utf8]{inputenc}
\else % if luatex or xelatex
  \usepackage{fontspec}
  \ifxetex
    \usepackage{xltxtra,xunicode}
  \fi
  \defaultfontfeatures{Mapping=tex-text,Scale=MatchLowercase}
  \newcommand{\euro}{€}
\fi
% use microtype if available
\IfFileExists{microtype.sty}{\usepackage{microtype}}{}
\usepackage[]{natbib}
\bibliographystyle{plainnat}

\usepackage{graphicx}
\ifxetex
  \usepackage[setpagesize=false, % page size defined by xetex
              unicode=false, % unicode breaks when used with xetex
              xetex]{hyperref}
\else
  \usepackage[unicode=true]{hyperref}
\fi
\hypersetup{breaklinks=true,
            bookmarks=true,
            pdfauthor={},
            pdftitle={Leveraging GTFS to explore spatial patterns in transit supply with respect to social needs},
            colorlinks=false,
            urlcolor=blue,
            linkcolor=magenta,
            pdfborder={0 0 0}}

\setcounter{secnumdepth}{5}
% Pandoc toggle for numbering sections (defaults to be off)


% tightlist command for lists without linebreak
\providecommand{\tightlist}{%
  \setlength{\itemsep}{0pt}\setlength{\parskip}{0pt}}




\usepackage{subfig}
\usepackage{booktabs}
\usepackage{longtable}
\usepackage{array}
\usepackage{multirow}
\usepackage{wrapfig}
\usepackage{float}
\usepackage{colortbl}
\usepackage{pdflscape}
\usepackage{tabu}
\usepackage{threeparttable}
\usepackage{threeparttablex}
\usepackage[normalem]{ulem}
\usepackage{makecell}
\usepackage{xcolor}



\begin{document}


\begin{frontmatter}

  \title{Leveraging GTFS to explore spatial patterns in transit supply
with respect to social needs}
    \author[Public Transport Research Group (PTRG)]{James Reynolds%
  %
  \fnref{1}}
   \ead{james.reynolds@monash.edu} 
    \author[Public Transport Research Group (PTRG)]{Yanda Qu%
  %
  \fnref{2}}
   \ead{yanda.qu@monash.edu} 
    \author[Public Transport Research Group (PTRG)]{Graham Currie%
  \corref{cor1}%
  \fnref{3}}
   \ead{graham.currie@monash.edu} 
      \affiliation[Public Transport Research Group (PTRG)]{
    organization={Public Transport Research Group (PTRG), Institute of
Transport Studies, Department of Civil Engineering Engineering, Monash
University},addressline={Clayton
Campus},city={Melbourne},postcode={3800},state={Victoria},country={Australia},}
    \cortext[cor1]{Corresponding author}
    \fntext[1]{Research Fellow}
    \fntext[2]{PhD Student}
    \fntext[3]{Professor}
  
  \begin{abstract}
  This is the abstract.

  It consists of two paragraphs.
  \end{abstract}
    \begin{keyword}
    keyword1 \sep 
    keyword2
  \end{keyword}
  
 \end{frontmatter}

\section{Introduction}\label{introduction}

A key reason for providing public transport services is to providing a
basic level of mobility for those who cannot otherwise drive themselves.
In many places, especially those with low-density landuse patterns
and/or limited transit servcies, it may be difficult or even impossible
to get around without a private vehicle. Disability, youth or old age,
socio-economic status, lack of a drivers license, and many other factors
might also mean someone is reliant on transit services for some or all
of their travel beyond their local neighbourhood. Adopting social
equity-focused approaches to transport planning might therefore suggest
targeting services towards spatial areas where there are higher social
and transport needs for transit.

Approaches for assessing spatial gaps between social needs and transit
supply have been reported in previous research including
\citet{Currie2003Hobart}, \citet{Currie2004Gap},
\citet{Currie2007Identifying} and \citet{currie2010identifying}. These
presented a transit Supply Index (SI), which was used to identify
geographic areas in a few Australian cities where there were very high
social needs for transport, but very little transit service or none at
all. This approach suggested a new way for planners to identify and
prioritise service provision towards locations and populations more
likely to be lacking sufficient mobility to meet their basic needs or
fully participate in society.

However, in almost two decades since the SI was introduced there does
not appear to have been much further use or development of this
approach. As well, it is unclear whether gaps between social needs and
transit supply have increased or reduced in Australian cities, and if
the spatial patterns identified in Hobart
\citep{Currie2003Hobart, Currie2004Gap} and Melbourne
\citep{Currie2007Identifying, currie2010identifying} are representative.
This may in part be because applying such methodologies elsewhere would,
at the time, likely have required bespoke data collection, cleaning and
analysis. \citet{Currie2003Hobart}, \citet{Currie2004Gap},
\citet{Currie2007Identifying} and \citet{currie2010identifying} had data
provided\\
direct from transit operators and authorities, but at the time transit
schedule data was typically only publically available in paper or PDF
format.\\
As well, the volume of spatial and service data and bespoke processing
needed to produce a social needs-gap analysis may have put such analyses
beyond reach in most places.

Nowadays, however, the General Transit Feed Specification (GTFS) and the
internet allows timetable data to be published in a standardized and
accessible format, and\\
more than 10,000 transit agencies publicly release data this way.
Various tools for manipulating such data are also now available,
including software for validating, analysing and visualizating GTFS, as
well as a separate standard (GTFS-Realtime) for sharing live vehicle
locations \citep{GTFS}. However, software tools for examining spatial
patterns and gaps in transit supply with respect to social needs for
transport appears limited.

This gap and the lack of direct follow up to \citet{Currie2003Hobart},
\citet{Currie2004Gap}, \citet{Currie2007Identifying} and
\citet{currie2010identifying} provide the motivation for the research
reported in this paper.\\
The overall aim of this research is to apply the developments in spatial
data science in the intervening years to social and transport needs-gaps
analysis. This encompasses three main objectives: (1) to develop tools
for needs-gap analysis using GTFS datasets; (2) to explore whether such
gaps have changed in the intervening years; and (3) to better understand
whether spatial patterns in Melbourne and Hobart are representative of
those in other places.

This research included the development of a new R package
(gtfssupplyindex) for calculating SI scores. Also presented in this
paper are results for Australian cities in 2016 and 2021, matching the
most recent censuses, allowing comparison to the 2006 analysis reported
in \citet{currie2010identifying} and between locations.

The remainder of this paper is structured as follows: the next section
outlines the background to this research, including the formulation of
the Transit Supply Index (SI). Section 3 describes the study
methodology, followed by presentation of results in Section 4 and
discussion in Section 5. Limitations of this study, directions for
future research and a brief conclusion are provided in Section 6.

\section{Background}\label{background}

\subsection{Transit metrics}\label{transit-metrics}

There are many metrics available for benchmarking transit services,
including: those in the Transit Cooperative Research Program (TCRP)
Report 88 guidebook on developing performance-measurement systems
\citep{Ryus:2003aa} and in the Transit Capacity and Quality of Service
Manual (TCQSM)\citep{TCQSM:2013}; those used across benchmarking
databases and programs (e.g.
\citet{Florida-Transit-Information-System:2018aa}, \citet{UITP:2015aa}
\citet{Imperial-College-London:2023aa}; and the results available on the
.Transit Score website \citep{WalkScore:2023tg}. The Fielding Triangle
\citep{FieldingGordonJ1987Mpts} provides a framework for combining
indicators of service inputs, outputs and consumption to describe cost
efficiency, cost effectiveness and service effectiveness. More broadly:
\citet{Litman:2003ab} and \citet{Litman:2016aa} discuss some of the
traffic, mobility, accessibility, social equity, strategic planning and
other rational decision-making-based perspectives underling transport
indicators; \citet{Reynolds:2017ah} extends these into models of how
institutionalism, incrementalism and other public policy analysis
concepts might apply to decision-making processes relating to transit
prioritization; \citet{GuzmanLuisA.2017Aeit} developed a measure of
accessibility in the context of policy development and social equity for
Latin American Bus Rapid Transit (BRT) networks; and
\citet{Creutzig2020streetspaceallocation} introduced street space
allocation metrics based around ten ethical principles.

However, many of these metrics may be difficult to calculate, explain or
understand, especially for those who are not planners, engineers or
other technical specialists. Where pre-calculated transit metrics are
immediately available, such as on a website or other online platform, it
may not be possible to independently generate scores, for instance to
assess proposed system changes. Contrasting examples are provided by the
metrics in the TCQSM and the Transit Score metric
\citep{WalkScore:2023tg}:

\begin{itemize}
\item
  Transit Scores are readily available on the Transit Score website for
  locations with a published GTFS feed. The meaning of these Transit
  Scores also appears easy to explain, with the highest possible score
  of 100 representing the sort of transit accessibility that might be
  xperienced in the center of New York. However, the Transit Score
  algorithm is a black box, and scores cannot be calculated
  independently or generated for proposed changes to networks.
\item
  the TCQSM provides a wide range of metrics for measuring different
  aspects of a transit system. The TCQSM scores themselves appear easy
  to understand or explain, ranging from A (good) to F (bad), although
  the large number of metrics might be somewhat overwhelming for some
  users. The scores, however, can be calculated independently, given
  sufficient data.
\end{itemize}

\citet{Wong:2013aa} provides an example of what can be done with GTFS
data, open metrics and coding by reporting the distribution of various
TCQSM metrics across 50 USA transit operators. Code used in the
\citet{Wong:2013aa} analysis is available for those who might wish to
produce a similar study for other locations and systems. Producing code
that allows similar calculations from GTFS data, except for the SI
metric, is included in the objectives of this study.

\subsection{The Transit Suppy Index}\label{the-transit-suppy-index}

A generalized form of the SI equation, adapted from
\citet{currie2010identifying}, is:

\[SI_{area, time} = \sum{\frac{Area_{Bn}}{Area_{area}}*SL_{n, time}}\]

where:

\begin{itemize}
\item
  \(SI_{area, time}\) is the Supply Index for the area of interest and a
  given period of time;
\item
  \(Area_{Bn}\) is the buffer area for each stop (n) within the area of
  interest (in \citet{currie2010identifying} this was based on a radius
  of 400 metres for bus and tram stops, and 800 metres for railway
  stations);
\item
  \(Area_{area}\) is the area of the area of interest; and
\item
  \(SL_{n,time}\) is the number of transit arrivals for each stop for a
  given time period.
\end{itemize}

\begin{figure}
\includegraphics[width=1\linewidth]{graphics/Currie2010SI} \caption{Distribution of supply measure scores – Metropolitan Melbourne (2006), Source: Currie (2010)}\label{fig:Currie_map_SI}
\end{figure}

\citet{currie2010identifying} reported SI scores for Census Collection
Districts (CCDs) across Greater Melbourne in 2006, as shown in Figure
\ref{fig:Currie_map_SI}, and identified a general pattern of: more
transit supply in the middle and inner suburbs and along passenger
railway lines; and outer areas tending to have very low SI scores or no
transit supply at all.

\subsection{Social need and needs gap}\label{social-need-and-needs-gap}

As well as measuring transit supply, \citet{currie2010identifying} also
assessed the social need for transit across Greater Melbourne using: the
Australian Bureaus of Statistics' Index of Related Socio-Economic
Advantage/Disadvantage (IRSAD) and a transport needs index derived from
eight weighted indicators. The spatial distribution of this composite
social needs index in 2006, reproduced in Figure
\textbackslash ref\{fig:Currie\_maps\_needs), indicates areas of above
average, high and very high social needs in 2006 were located in: some
outer areas, particularly in the east and south-east; and in some middle
areas in the south-east, north and west. \citet{currie2010identifying}
also identified areas with very high transport needs, but very low or no
transit supply,\\
as reproduced in Figure \ref{fig:Currie_map_gap}. This indicated areas
where service gaps might be of particular concern, Most of these were
located in outer parts of Melbourne in the north-east, south-east and
south, although there were also some pockets in the middle suburbs in
the west, north and south east.

\begin{figure}
\includegraphics[width=1\linewidth]{graphics/Currie2010Needs} \caption{Distribution of categories of composite social need index scores in 2006, Source: Currie (2010)}\label{fig:Currie_map_needs}
\end{figure}

\begin{figure}
\includegraphics[width=1\linewidth]{graphics/Currie2010gap} \caption{Melbourne needs-gap in 2006 – very high transport need areas with zero or very low public transport supply, Source: Currie (2010)}\label{fig:Currie_map_gap}
\end{figure}

Overall \citet{currie2010identifying} found that ``8.2\% of Melbourne
residents have `very high' needs but `zero', `low' or `very low' public
transport supply.'' Using this methodology in transit planning appraoch
was also stated as being ``substantially more useful than the
presentation of anecdotal evidence, which is the most common means of
identifying transport needs in local transport studies throughout the
world.''\citep{currie2010identifying}

However, it does not appear that this approach has been widely adopted
in practice or further developed by researchers. Our suspicion is that
while the SI has a relatively simple formula and requires only
geographic and timetable data, the lack of a software tool to perform
these calculations may be part of the reason that it has not been more
widely adopted. It is also unclear whether the patterns in Melbourne
identified in \citet{currie2010identifying}, have changed since the 2006
analysis. Developing a software tool to calculate SI tools from GTFS
data, and then using it to comparing current conditions and other
locations to the findings of \citet{currie2010identifying}, therefore,
provides motivation for this research.

\section{Methodology}\label{methodology}

\subsection{Code development}\label{code-development}

This study developed a package of tools for calculating the SI from GTFS
data using the R programming language \citep{R-base}. The
recommendations of \citet{wickham2023r} informed the package setup and
development approach. Various existing packages and code examples were
relied upon including: the sf package \citep{R-sf} for geospatial
analysis; the tidyverse \citep{tidyverse2019}; gtfstools
\citep{R-gtfstools}; and tidytransit \citep{R-tidytransit}. Australian
Bureau of Statistics (ABS) data was also used, sourced via the strayr
and absmapsdata packages \citep{r-strayr}.

Code was developed and tested on the Mornington Peninsula Tourist
Railway GTFS feed. This was selected primarily for convenience, given
that the authors are familiar with the surrounding geography and that
the feed covers a small number of trips across just three stations.

Statistical Area 1 (SA1) zones\\
were adopted from the Australian Bureau of Statistics \citep{ABSmaps} as
the areas of interest. SA1 zones are the smallest geographical areas for
which results are reported in the Australian census. SA3 zones, which
are generally similar to Local Government Area (LGA) boundaries, where
used when aggregating.

\subsection{Changes since 2006: Greater
Melbourne}\label{changes-since-2006-greater-melbourne}

Much has changed since 2006, including the spatial geography used by the
Australian Bureau of Statistics (ABS) to collect census data. To allow
direct comparison between 2006 and now, therefore, this study first
calculated SI scores for the week starting the day of the 2021 census
using the same Census Collection Districts (CCDs) used by
\citet{currie2010identifying}. The Victorian GTFS feed, published by
Public Transport Victoria (PTV), was used, with historical feeds sourced
via \citet{transitfeeds_victoria:2023aa}.

Unfortunately, it is not possible to obtain 2016 or 2021 social
disadvantage data for CCDs, as the ABS now releases data for SA zones
only. SA1 zones, therefore, are adopted for the comparison of social
needs-gaps in 2016 and 2021, as discussed in the following.

\subsection{Variation in spatial patterns across
location.}\label{variation-in-spatial-patterns-across-location.}

SI scores were also calculated\\
for the weeks starting the day of the 2016 and 2021 censuses in Greater
Melbourne, Greater Brisbane, Greater Perth, Greater Adelaide and Greater
Hobart. Historical GTFS data was again sourced via the Transit Feeds
website, Unfortunately it was not possible to locate historical data for
Greater Sydney, so the latest data set was sourced directly from
Transport for NSW. SA1 zones were adopted as the areas of interest so as
to allow direct comparison to ABS-reported values for social needs,
transport needs, population and other census data.

\subsection{Measuring social
disadvantage}\label{measuring-social-disadvantage}

This study adopts a similar approach to measuring social disadvantage as
used in \citet{currie2010identifying}, using: the ABS' Index of Relative
Socio-Economic Advantage/Disadvantage (IRSAD); and a transport needs
index\footnote{The same need indicators and weightings used in
  \citet{currie2010identifying} were adopted, although \$799 or lower
  per week was used as the threshold for low income households rather
  than \$499 to account for inflation (as per Reserve Bank of
  Australia's online inflation calculator).}. A composite needs
indicator was derived based on the IRSAD and the transport needs index,
again as per the \citet{currie2010identifying} approach\footnote{\hfill\break
  However, changes to the ABS reporting systems mean that the composite
  needs indicator had to based on weighting both the IRSAD index and the
  transport need index by the total population of each SA1 zone, which
  were then added, standardised and split into six groups.}.

\section{Results}\label{results}

\subsection{The gtfssupplyindex
Package}\label{the-gtfssupplyindex-package}

Code developed to calculate SI scores is available as an R package on
github as \citet{gtfssupplyindex_github}. Included in the package is a
vignette that outlines the structure of the calculations, the developed
functions (LINK HERE), The vignette also includes step-by-step
calculations for the Mornington Peninsula Railway that provide a worked
example and comparison to SI scores calculated manually.

\subsection{Changes since 2006: Greater
Melbourne}\label{changes-since-2006-greater-melbourne-1}

Figure \ref{fig:Greater_Melbourne_CCD_2021} shows SI scores\\
for Melbourne in the week of the 2021 census and using the same 2006 CCD
boundaries as in Figure \ref{fig:Currie_map_SI}.

\subsection{Variation in spatial patterns across Australian
cities}\label{variation-in-spatial-patterns-across-australian-cities}

\textless NEED TO ADD OTHER CITIES. MIGHT BE BEST TO CHANGE ALL OF THE
NEXT CHUCK INTO A FUNCTION TO THEN APPLY TO EACH CITY IN
TURN\textgreater{}

\begin{figure}
\centering
\includegraphics{Leveraging_GTFS_to_assess_transit_supply_Transport_Geography_files/figure-latex/Australian_cities_2021-1.pdf}
\caption{Transit Supply by SA1, weeks starting the date of 2021 census}
\end{figure}

Figure \ref{fig:Australian_cities_2021} shows SI values for the week
starting on the day of the 2021 census day for Greater Melbourne in 2016
and 2021. Comparison to Figure \ref{fig:Currie_map_SI} indicates that.
The transit network and the Greater Melbourne Greater Capital City
Statistical Area (GCCSA) both cover more area than they did in 2006.
However, there are still many areas, especially in the outer suburbs,
with very low or zero transit supply.

Comparing 2016 and 2021 suggests that the geographic spread of service
levels has remained largely the same, albeit with some areas that had
little to no transit service appearing to have new services.

\begin{tabular}{l|l|l}
\hline
**Characteristic** & **2016**, N = 10,723 & **2021**, N = 10,999\\
\hline
SI & NA & NA\\
\hline
Mean (SD) & 489 (1,019) & 486 (982)\\
\hline
Median (25\%,75\%) & 229 (92,524) & 237 (95,523)\\
\hline
Range & 0, 15,781 & 0, 15,346\\
\hline
\end{tabular}

\begin{figure}
\centering
\includegraphics{Leveraging_GTFS_to_assess_transit_supply_Transport_Geography_files/figure-latex/Greater_Melbourne_2016_2021_year_SI-1.pdf}
\caption{SI scores by SA3, census day 2016 and 2021}
\end{figure}

Figure \ref{fig:Greater_Melbourne_2016_2021_year_SI} compares SI scores
for Greater Melbourne across 2016 and 2021. Results indicate a
statistically significant difference, but there does not appear to have
been much change across median, mean, interquartile range or overall
range (See table)

Figure \ref{fig:Greater_Melbourne_2016_2021_scatterplot} compares the
2016 and 2021 SI scores for SA1 zones across Greater Melbourne. Results
indicate a statistically significant relationship, and for most SA1
zones it appears that SI scores have remained largely the same in 2021
as they were in 2016. Of interest, however, may be a group of SA1s that
had SI scores in the range of 3-4,000 in 2016, but only around 2,500 in
2021.

\citet{currie2010identifying} also reported the distribution of SI
scores by category across the number of CCDs and the resident
population.

\begin{table}

\caption{\label{tab:Greater_Melbourne_2016_2021_sa1_population}Distribution of supply index categories, by number of SA1 zones for 2021}
\centering
\begin{tabular}[t]{l|r|l|l}
\hline
SI\_binned & n & percent & valid\_percent\\
\hline
Zero & 0 & 0.0\% & 0.0\%\\
\hline
Very low & 1803 & 16.4\% & 16.4\%\\
\hline
Low & 1794 & 16.3\% & 16.3\%\\
\hline
Below median & 1838 & 16.7\% & 16.7\%\\
\hline
Above median & 1888 & 17.2\% & 17.2\%\\
\hline
High & 1851 & 16.8\% & 16.8\%\\
\hline
Very high & 1821 & 16.6\% & 16.6\%\\
\hline
NA & 4 & 0.0\% & -\\
\hline
Total & 10999 & 100.0\% & 100.0\%\\
\hline
\end{tabular}
\end{table}

\subsubsection{Social needs}\label{social-needs}

Figure \ref{fig:Greater_Melbourne_2021_social_needs} shows the
distribution of categories of social need index scores across Greater
Melbourne for 2021. This figure is analogous to the 2006 value from
\citet{currie2010identifying} shown in Figure \ref{fig:Currie_map_needs}
although, as discussed in the methodology section above, it was not
possible to exactly replicate the \citet{currie2010identifying} approach
as the total number of people within one or more of social need
component groups (necessary to calculate the two relative need
indicators) are not reported in the 2021 census.

\includegraphics{Leveraging_GTFS_to_assess_transit_supply_Transport_Geography_files/figure-latex/Greater_Melbourne_2021_social_needs-1.pdf}
Figure \ref{fig:Greater_Melbourne_2021_social_needs} appears to indicate
that there is no clear spatial pattern to the distribution of the
categories of the composite need index scores. This appears to contrast
to trend, albeit with some exceptions, towards very high social needs
scores in outer areas identified by \citet{currie2010identifying}
(Figure \ref{fig:Currie_map_needs}). This may, however, be an artifact
of either the differences in the composite needs scores used in this
analysis (due to the lack of data to assess relative needs) compared to
the \citet{currie2010identifying} analysis. As well, the 2021 census SA1
zones generally appear to smalller than the 2006 Census Collection
Districts (CCDs) used in \citet{currie2010identifying}, especially in
outer areas. This will be associated with the growth of Greater
Melbourne's population and spatial dispersment, with many of the large
outer `Very High' CCDs shown in the 2006 map now split into many more
SA1 zones.

\subsubsection{Needs-gap analysis}\label{needs-gap-analysis}

\begin{verbatim}
##  [1] Very high     High          Above average Below average Low          
##  [6] Very low      Very high     High          Above average Below average
## [11] Low           Very low      Very high     High          Above average
## [16] Below average Low           Very low      Very high     High         
## [21] Above average Below average Low           Very low      Very high    
## [26] High          Above average Below average Low           Very low     
## [31] Very high     High          Above average Below average Low          
## [36] Very low      Very high     High          Above average Below average
## [41] Low           Very low      Very high     High          Above average
## [46] Below average Low           Very low      Very high     High         
## [51] Above average Below average Low           Very low      Very high    
## [56] High          Above average Below average Low           Very low     
## [61] Very high     High          Above average Below average Low          
## [66] Very low      Very high     High          Above average Below average
## [71] Low           Very low      Very high     High          Above average
## [76] Below average Low           Very low      Very high     High         
## [81] Above average Below average Low           Very low     
## Levels: Very low Low Below average Above average High Very high
\end{verbatim}

\begin{figure}
\centering
\includegraphics{Leveraging_GTFS_to_assess_transit_supply_Transport_Geography_files/figure-latex/Greater_Melbourne_2006_2021_needs_gap_zones-1.pdf}
\caption{Number of areas by SI and social/transport need category,
comparison between 2006 and 2021}
\end{figure}

\begin{figure}
\centering
\includegraphics{Leveraging_GTFS_to_assess_transit_supply_Transport_Geography_files/figure-latex/Greater_Melbourne_2006_2021_needs_gap_population-1.pdf}
\caption{Population by SI and social/transport need category, comparison
between 2006 and 2021}
\end{figure}

Figure \ref{fig:Greater_Melbourne_2006_2021_needs_gap_zones} compares
the number of zones\footnote{SA1s in 2021, versus CCD in 2006.} by SI
and social/transport need category between 2006 and 2021. Population is
compared in Figure
\ref{fig:Greater_Melbourne_2006_2021_needs_gap_population}

These results indicate that, in 2021, 66 SA1 zones, (representing
0.6\%of the total 11,138SA1 zones in Greater Melbourne) had no public
transport, but very `very high' social/transport needs. These zones have
a combined population of

\ensuremath{4.5074\times 10^{4}} SA1 zones, representing 45,074(0.9\% of
the total 4,885,773 population) live in areas with no public transport
but have `very high' social/transport needs. This compares to the 89
CCDs (1.6\% of the 5,720 total), representing 37,699 Melbourne residents
(1.1\% of the population) living in areas with no public transport, but
very high social/transport needs reported for 2006 in
\citet{currie2010identifying}.

\subsection{Variation in spatial patterns across time: 2016 and
2021}\label{variation-in-spatial-patterns-across-time-2016-and-2021}

INCLUDE GINI COEFFICIENT PLOTS HERE

\section{Discussion}\label{discussion}

\subsection{Limitations}\label{limitations}

\subsection{Directions for furture
research}\label{directions-for-furture-research}

\section{Conclusions}\label{conclusions}

\renewcommand\refname{References}
\bibliography{References.bib, packages.bib}


\end{document}
