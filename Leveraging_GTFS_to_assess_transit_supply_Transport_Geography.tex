\documentclass[preprint, 3p,
authoryear]{elsarticle} %review=doublespace preprint=single 5p=2 column
%%% Begin My package additions %%%%%%%%%%%%%%%%%%%

\usepackage[hyphens]{url}

  \journal{Transport Geography?} % Sets Journal name

\usepackage{graphicx}
%%%%%%%%%%%%%%%% end my additions to header

\usepackage[T1]{fontenc}
\usepackage{lmodern}
\usepackage{amssymb,amsmath}
% TODO: Currently lineno needs to be loaded after amsmath because of conflict
% https://github.com/latex-lineno/lineno/issues/5
\usepackage{lineno} % add
\usepackage{ifxetex,ifluatex}
\usepackage{fixltx2e} % provides \textsubscript
% use upquote if available, for straight quotes in verbatim environments
\IfFileExists{upquote.sty}{\usepackage{upquote}}{}
\ifnum 0\ifxetex 1\fi\ifluatex 1\fi=0 % if pdftex
  \usepackage[utf8]{inputenc}
\else % if luatex or xelatex
  \usepackage{fontspec}
  \ifxetex
    \usepackage{xltxtra,xunicode}
  \fi
  \defaultfontfeatures{Mapping=tex-text,Scale=MatchLowercase}
  \newcommand{\euro}{€}
\fi
% use microtype if available
\IfFileExists{microtype.sty}{\usepackage{microtype}}{}
\usepackage[]{natbib}
\bibliographystyle{plainnat}

\ifxetex
  \usepackage[setpagesize=false, % page size defined by xetex
              unicode=false, % unicode breaks when used with xetex
              xetex]{hyperref}
\else
  \usepackage[unicode=true]{hyperref}
\fi
\hypersetup{breaklinks=true,
            bookmarks=true,
            pdfauthor={},
            pdftitle={Leveraging GTFS data to assess transit supply},
            colorlinks=false,
            urlcolor=blue,
            linkcolor=magenta,
            pdfborder={0 0 0}}

\setcounter{secnumdepth}{5}
% Pandoc toggle for numbering sections (defaults to be off)


% tightlist command for lists without linebreak
\providecommand{\tightlist}{%
  \setlength{\itemsep}{0pt}\setlength{\parskip}{0pt}}




\usepackage{booktabs}
\usepackage{longtable}
\usepackage{array}
\usepackage{multirow}
\usepackage{wrapfig}
\usepackage{float}
\usepackage{colortbl}
\usepackage{pdflscape}
\usepackage{tabu}
\usepackage{threeparttable}
\usepackage{threeparttablex}
\usepackage[normalem]{ulem}
\usepackage{makecell}
\usepackage{xcolor}



\begin{document}


\begin{frontmatter}

  \title{Leveraging GTFS data to assess transit supply}
    \author[Public Transport Research Group (PTRG)]{James Reynolds%
  \corref{cor1}%
  \fnref{1}}
   \ead{james.reynolds@monash.edu} 
    \author[Public Transport Research Group (PTRG)]{Yanda Qu%
  %
  \fnref{2}}
   \ead{yanda.qu@monash.edu} 
    \author[Public Transport Research Group (PTRG)]{Graham Currie%
  %
  \fnref{3}}
   \ead{graham.currie@monash.edu} 
      \affiliation[Public Transport Research Group (PTRG)]{
    organization={Institute of Transport Studies, Department of Civil
Engineering Engineering, Monash University},addressline={Clayton
Campus},city={Melbourne},postcode={3800},state={Victoria},country={Australia},}
    \cortext[cor1]{Corresponding author}
    \fntext[1]{Research Fellow}
    \fntext[2]{PhD Strudent}
    \fntext[2]{Professor}
  
  \begin{abstract}
  This is the abstract.

  It consists of two paragraphs.
  \end{abstract}
    \begin{keyword}
    keyword1 \sep 
    keyword2
  \end{keyword}
  
 \end{frontmatter}

\hypertarget{introduction}{%
\section{Introduction}\label{introduction}}

\citet{Berenson2016} highlights that the quote ``if you can't measure
it, you can't manage it'' is often miss-attributed to
\citet{Deming1993new}, and that he was actually trying to make the
opposite point in \emph{If you can't measure performance, can you
improve it?}. Regardless, service level indicators are an important part
of researching, managing and seeking to improve transit operations
\citep{FieldingGordonJ1987Mpts, Ryus:2003aa}. A wide range of indicators
already, including, for example: those in the Transit Capacity and
Quality of Service Manual (TCQSM)\citep{TCQSM:2013}, the Transit Score
metric \citep{WalkScore:2023tg} and many more.

Practitioners, researchers and advocates seeking to use such metrics may
face two inter-related challenges: (1) calculating the metrics
themselves for a specific location and service pattern; and (2)
explaining the metrics, their meaning and importance to those who might
not be specialists in transit, such as to politicians or the general
public. For example, the TCQSM metrics appear difficult to calculate in
practice without access to specialist software and data. They appear
relatively easy to explain given they use an A to F scoring system and
there is an entire guidebook about them. However, although this may be
offset by the multitude of different indicators. In contrast, Transit
Scores can be obtained simply by typing an address into a website, but
cannot be calculated independently as the methodolgy / algorithm is not
publically available.

Previous research by \citet{currie2007identifying} developed a transit
Supply Index (SI) metric that is both relatively easy to calculate and
explain. It is obtained by calculating the number of transit arrivals at
stops within an area of interest, with an adjustment made to account for
the typical walk-access catchment for each stop. Hence, higher SI scores
indicate areas with higher frequency and/or better coverage.

Unfortunately, the SI does not appear to have been widely used, perhaps
in part because at the time it was first published timetable data was
not publicly available in a standardized and machine-readable format.
The scores reported in Currie and Senbergs (2007) were calculated
directly from a database of services provided by the transit authority
in Melbourne, Australia. Since then, however, the General Transit Feed
Specification (GTFS) has been developed as a way to publish timetable
data, along with many processing tools. More than 10,000 agencies are
now providing GTFS feeds\footnote{There are two forms: GTFS-static
  consisting of the timetable data (the scheduled services); and
  GTFS-realtime, which includes vehicle arrivals and departure times
  based on real-world position data. This paper and project uses only
  the GTFS-static (timetable) format.} \citep{GTFS},

A gap, however, is that a tool to calculate SI scores directly from a
GTFS dataset has not yet been available. This provides the motivation in
this paper, in which a new R package (gtfssupplyindex) specifically
developed to calculate SI scores is presented.

\renewcommand\refname{References}
\bibliography{References.bib}


\end{document}
