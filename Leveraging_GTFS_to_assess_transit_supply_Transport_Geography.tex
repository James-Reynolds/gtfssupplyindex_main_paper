\documentclass[preprint, 3p,
authoryear]{elsarticle} %review=doublespace preprint=single 5p=2 column
%%% Begin My package additions %%%%%%%%%%%%%%%%%%%

\usepackage[hyphens]{url}

  \journal{Transport Geography?} % Sets Journal name

\usepackage{graphicx}
%%%%%%%%%%%%%%%% end my additions to header

\usepackage[T1]{fontenc}
\usepackage{lmodern}
\usepackage{amssymb,amsmath}
% TODO: Currently lineno needs to be loaded after amsmath because of conflict
% https://github.com/latex-lineno/lineno/issues/5
\usepackage{lineno} % add
\usepackage{ifxetex,ifluatex}
\usepackage{fixltx2e} % provides \textsubscript
% use upquote if available, for straight quotes in verbatim environments
\IfFileExists{upquote.sty}{\usepackage{upquote}}{}
\ifnum 0\ifxetex 1\fi\ifluatex 1\fi=0 % if pdftex
  \usepackage[utf8]{inputenc}
\else % if luatex or xelatex
  \usepackage{fontspec}
  \ifxetex
    \usepackage{xltxtra,xunicode}
  \fi
  \defaultfontfeatures{Mapping=tex-text,Scale=MatchLowercase}
  \newcommand{\euro}{€}
\fi
% use microtype if available
\IfFileExists{microtype.sty}{\usepackage{microtype}}{}
\usepackage[]{natbib}
\bibliographystyle{plainnat}

\usepackage{graphicx}
\ifxetex
  \usepackage[setpagesize=false, % page size defined by xetex
              unicode=false, % unicode breaks when used with xetex
              xetex]{hyperref}
\else
  \usepackage[unicode=true]{hyperref}
\fi
\hypersetup{breaklinks=true,
            bookmarks=true,
            pdfauthor={},
            pdftitle={Leveraging GTFS data to assess transit supply},
            colorlinks=false,
            urlcolor=blue,
            linkcolor=magenta,
            pdfborder={0 0 0}}

\setcounter{secnumdepth}{5}
% Pandoc toggle for numbering sections (defaults to be off)


% tightlist command for lists without linebreak
\providecommand{\tightlist}{%
  \setlength{\itemsep}{0pt}\setlength{\parskip}{0pt}}




\usepackage{subfig}
\usepackage{booktabs}
\usepackage{longtable}
\usepackage{array}
\usepackage{multirow}
\usepackage{wrapfig}
\usepackage{float}
\usepackage{colortbl}
\usepackage{pdflscape}
\usepackage{tabu}
\usepackage{threeparttable}
\usepackage{threeparttablex}
\usepackage[normalem]{ulem}
\usepackage{makecell}
\usepackage{xcolor}



\begin{document}


\begin{frontmatter}

  \title{Leveraging GTFS data to assess transit supply}
    \author[Public Transport Research Group (PTRG)]{James Reynolds%
  \corref{cor1}%
  \fnref{1}}
   \ead{james.reynolds@monash.edu} 
    \author[Public Transport Research Group (PTRG)]{Yanda Qu%
  %
  \fnref{2}}
   \ead{yanda.qu@monash.edu} 
    \author[Public Transport Research Group (PTRG)]{Graham Currie%
  %
  \fnref{3}}
   \ead{graham.currie@monash.edu} 
      \affiliation[Public Transport Research Group (PTRG)]{
    organization={Public Transport Research Group (PTRG), Institute of
Transport Studies, Department of Civil Engineering Engineering, Monash
University},addressline={Clayton
Campus},city={Melbourne},postcode={3800},state={Victoria},country={Australia},}
    \cortext[cor1]{Corresponding author}
    \fntext[1]{Research Fellow}
    \fntext[2]{PhD Strudent}
    \fntext[3]{Professor}
  
  \begin{abstract}
  This is the abstract.

  It consists of two paragraphs.
  \end{abstract}
    \begin{keyword}
    keyword1 \sep 
    keyword2
  \end{keyword}
  
 \end{frontmatter}

\hypertarget{introduction}{%
\section{Introduction}\label{introduction}}

The Transit Score website \citep{WalkScore:2023tg} will return a score
out of 100 reflecting the quantity of transit available.\\
This score is easy to obtain and simple to explain to a non-technical
audience. However, the score cannot be calculated independently, which
might limit practitioners, researchers, advocates or others involved in
transit planning, operations or policy-making from reporting score
changes associated with new infrastructure, service patterns or
improvement options.

In contrast, the Transit Capacity and Quality of Service Manual
(TCQSM)\citep{TCQSM:2013} provides a 700 page guidebook full of
indicators, all of which can be calculated independently if the
necessary data is available. However, the TCQSM and many other transit
service indicators developed in research and practice are focused
towards the needs of practitioners and researchers. There is no simple
`overall' metric that might be easily obtained, used and/or understood
by those who are not engineers, planners or other technically-minded
people.

Previous research by \citet{currie2007identifying} developed a transit
Supply Index (SI), which has the advantage of providing a single score.
The SI score is based on the number of transit arrivals at stops within
an area of interest,\\
with an adjustment made to account for how much of the area of interest
is within walking distance of each stop. Unfortunately, the SI does not
appear to have been widely used, perhaps in part because at the time it
was first published timetable data was not typically available in a
standardized and machine-readable format.\\
Nowadays, the General Transit Feed Specification (GTFS) allows timetable
publication in a standardized format, with more than 10,000 agencies
providing feeds\citep{GTFS}. Many visualization, processing and analysis
tools that accept GTFS data are now available, and it is this data that
provides the input necessary for the Transit Score website to return a
result.

A gap, however, is that there is not yet a tool to calculate SI scores
directly from GTFS datasets. This provides the motivation for the
research reported in this paper, in which a new R package
(gtfssupplyindex) specifically developed to calculate SI scores is
presented. The remainder of this paper is structured as follows: the
next section outlines the background to this research, including the
original formulation of the Transit Supply Index, and an explanation of
the GTFS. Section 3 then describes the study methodology, followed by a
brief presentation of results in Section 4. Section 5 discusses the
results, outlines directions for future research and provides a brief
conclusion.

\hypertarget{background}{%
\section{Background}\label{background}}

\hypertarget{transit-metrics}{%
\subsection{Transit metrics}\label{transit-metrics}}

Even a brief search reveals many metrics available for benchmarking
transit services. Examples include: (1) those in the Transit Cooperative
Research Program (TCRP) Report 88, which is an extensive guidebook on
developing a performance-measurement system \citep{Ryus:2003aa}; (2)
online databases provided by the Florida Transit Information System
(FTIS) \citep{Florida-Transit-Information-System:2018aa} and
\citet{UITP:2015aa}; (3) those used in the extensive annual benchmarking
program undertaken yearly by the Transport Strategy Centre in the United
Kingdom, including over 100 transit providers around the world
\citep{Imperial-College-London:2023aa}; and (4) a recently developed
methodology to calculate `blank spots', beyond typical walking access
distances to/from transit stops\citep{AlamriSultan2023GAoA}.

The Fielding Triangle \citep{FieldingGordonJ1987Mpts} provides a
framework for combining indicators of service inputs, outputs and
consumption to describe cost efficiency, cost effectiveness and service
effectiveness. More broadly:

\begin{itemize}
\item
  \citet{Litman:2003ab} and \citet{Litman:2016aa} discuss some of the
  traffic, mobility, accessibility, social equity, strategic planning
  and other rational decision-making-based perspectives underling
  transport indicators;
\item
  \citet{Reynolds:2017ah} extends these into models of how
  institutionalism, incrementalism and other public policy analysis
  concepts might apply to decision-making processes relating to transit
  prioritization;
\item
  \citet{GuzmanLuisA.2017Aeit}, developed a measure of accessibility in
  the context of policy development and social equity for Latin American
  Bus Rapid Transit (BRT) networks; and
\item
  \citet{Creutzig2020streetspaceallocation} introduced street space
  allocation metrics based around 10 ethical principles
\end{itemize}

However, many of these metrics appear difficult to calculate, complex to
explain or understand, and likely not well suited to communication with
those who are not planners or engineers, or other technical specialists.
Where pre-calculated metrics are immediately available it may not be
possible for practitioners, researchers or advocates to independently
generate metrics for proposed system changes. Sometimes it is not even
possible to know precisely how scores for the existing services levels
are calculated. For example, Transit Scores for locations with a
published GTFS feed are readily available on the
\citet{WalkScore:2023tg} website, eliminating the need for any
calculations. The meaning of these Transit Scores appears easy to
explain, as the highest possible score of 100 represents what might be
experienced in the centre of New York. However, the Transit Score
algorithm is patented and effectively a black box. Transit Scores cannot
be calculated independently or generated for proposed changes to
networks.

\hypertarget{gtfs}{%
\subsection{GTFS}\label{gtfs}}

The General Transit Feed Specification (GTFS) is an open, text-based
format developed originally to allow transit to be included in the
Google Maps navigation platform \citep{GTFS}. Figure @ref(fig:GTFS\_ERD)
shows an Entity Relationship Diagram (ERD) of the GTFS data structure.
This indicates how GTFS data is stored as a series of tables (agency,
routes, trips etc.) with primary and foreign keys (agency\_id,
route\_id, trip\_id etc.) providing links.

\begin{figure}
\includegraphics[width=1\linewidth]{graphics/GTFS} \caption{GTFS entity relationship diagram. Source: adapted by author from Alamri et al (2023) and the GTFS Schedule Reference (16/11/2023 revision).}\label{fig:GTFS_ERD}
\end{figure}

GTFS allows individual transit systems to be included in many online
products and analysis, including the Transit Score metric itself.
\citet{Wong:2013aa} provides another example of what can be done with
GTFS data, having developed code to calculate of some of the TCQSM
metrics and compared these across 50 transit operators.

\hypertarget{the-transit-suppy-index}{%
\subsection{The Transit Suppy Index}\label{the-transit-suppy-index}}

A generalized form of the Transit Supply Index (SI) from Currie and
Senbergs (2007) is:

\[SI_{area, time} = \sum{\frac{Area_{Bn}}{Area_{area}}*SL_{n, time}}\]

where: (1) \(SI_{area, time}\) is the Supply Index for the area of
interest and a given period of time; (2) \(Area_{Bn}\) is the buffer
area for each stop (n) within the area of interest. In Currie and
Senbergs (2007) this was based on a radius of 400 metres for bus and
tram stops, and 800 metres for railway stations; (3) \(Area_{area}\) is
the area of the area of interest; and (4) \(SL_{n,time}\) is the number
of transit arrivals for each stop for a given time period.

The SI score does not incorporate service span, speed or other elements
of a transit service. While these can be important to passenger
experience, they might add complexity. Simplicity is also helped by the
way that the SI is additive, in that \(SI_{area, time}\) scores can be
aggregated to calculate an overall score across multiple time periods or
for a region encompassing multiple areas of interest.

\hypertarget{methodology}{%
\section{Methodology}\label{methodology}}

This study developed a package with tools for calculating the SI from
GTFS data. The R programming language \citep{R-base} was adopted for
code development. Package development setup and workflow as described by
\citet{wickham2023r} was adopted. Various existing packages were relied
upon including: the sf package \citep{R-sf} for geospatial analysis; the
tidyverse \citep{tidyverse2019}; gtfstools \citep{R-gtfstools}; and
tidytransit \citep{R-tidytransit}. Some code was adapted from examples,
vignettes and other documentation in the tidytransit, gtfstools and
other packages.

Two cases where used during the code development and testing, such that
results might be generated for real GTFS data: the Mornington Peninsula
Tourist Railway GTFS feed and the Public Transport Victoria (PTV) GTFS
feed, both in Victoria, Australia. Both were selected primarily for
convenience, given that the authors are familiar with the typical
service patterns and geography.

Figure @ref(Melbourne\_map)) shows the areas of interest relevant to the
code development and testing, and selected railway stations. Statistical
Area (SA) zones from the Australian Bureau of Statistics \citep{ABSmaps}
Areas of interest included Greater Melbourne and its SA3 zones (main),
SA1 zones in the central part of Melbourne (top-right) and SA1 zones
within 800 metres of the Mornington Penninsula railway (bottom-right).

\begin{figure}
\includegraphics[width=1\linewidth]{graphics/all_maps} \caption{Areas of interest}\label{fig:Melbourne_map}
\end{figure}

\hypertarget{mornington-penninsula-tourist-railway}{%
\subsection{Mornington Penninsula Tourist
Railway}\label{mornington-penninsula-tourist-railway}}

The Morning Peninsula Tourist Railway is in the outer south-east of
Melbourne, running on Sundays and Wednesdays between Mornington and
Moorooduc, with an intermediate stop at Tanti Park (see
https://transitfeeds.com/p/mornington-railway/806/latest/stops). A GTFS
feed from 2018 was selected for the purposes of tests and demonstrating
the code and output. Australian Bureau of Statistics (ABS) data was also
used, sources via the strayr and absmapsdata packages \citep{r-strayr}.
The Mornington Peninsular Statistical Area 3 (SA3) zone and the
Statistical Area 1 (SA1) zones contained within it were adopted as the
areas of interest.

\hypertarget{public-transport-victoria-ptv}{%
\subsection{Public Transport Victoria
(PTV)}\label{public-transport-victoria-ptv}}

Larger scale testing was performed using the Victorian GTFS feed,
published by Public Transport Victoria (PTV) and sourced via
\citet{transitfeeds_victoria:2023aa} for historical feeds.

\hypertarget{results}{%
\section{Results}\label{results}}

\hypertarget{code-structure}{%
\subsection{Code structure}\label{code-structure}}

Developed code is available and documented on github
\citep{gtfssupplyindex_github}. The structure of the package, functions
developed, and data tables are shown in Figure @ref(fig:SI\_ERD). This
shows\\
how the package takes input from three files: a gtfs feed (gtfs.zip); a
sf object describing the geometry of the areas for which the SI is to be
calculated; and a csv file (included in the package) defining the buffer
zone distances for each route type. The ultimate output is a
si\_by\_area\_and\_hour table (bottom-right), which reports the SI score
for each hour of the day across dates specified by the user.

\begin{figure}
\includegraphics[width=1\linewidth]{graphics/SI_data_structure} \caption{Entity Relationship Diagram (ERD) showing the data structure and functions related to the gtfssupplyindex package}\label{fig:SI_ERD}
\end{figure}

\hypertarget{mornington-pennisula-tourist-railway}{%
\subsection{Mornington Pennisula Tourist
Railway}\label{mornington-pennisula-tourist-railway}}

Various functions and their output are explained in the following, using
the Mornington Peninsula GTFS for December 30th, 2018, and SA1 zone
boundaries as a worked example. Individual steps are:

\begin{enumerate}
\def\labelenumi{(\arabic{enumi})}
\item
  loading the gtfs.zip file: the gtfs\_by\_route\_type function loads
  the gtfs data and splits it into a list (by route\_type) of tidygtfs
  objects, using the filter\_by\_route\_type function from the gtfstools
  package \citep{filter_GTFS_by_mode}.
\item
  loading geometry information about the areas of interest: geographical
  data about the areas of interest are loaded by the
  load\_areas\_of\_interest.R function into an sf object, using the sf
  package \citep{R-sf}. The resultant areas\_of\_interest table contains
  each area\_id and its associated geometry. Data about buffer zones,
  specifically the walking distance threshold assigned to each
  route\_type (mode) is then loaded, again through a function
  (load\_buffer\_zone.R).
\item
  calculating which stops are within the catchment walking distance of
  which areas: using the stops\_in\_walk\_dist function. Figure
  @ref(fig:calculate\_stop\_in\_or\_near\_areas\_verbose)) shows how
  this function identified SA1 areas within the 800 metre catchment of
  the three Mornington stations.
\end{enumerate}

\begin{figure}
\includegraphics[width=1\linewidth]{Leveraging_GTFS_to_assess_transit_supply_Transport_Geography_files/figure-latex/calculate_stop_in_or_near_areas_verbose-1} \caption{Step 3, stop catchments for the Mornington Penninsula Tourist Railway, showing intersections with SA1 zones}\label{fig:calculate_stop_in_or_near_areas_verbose}
\end{figure}

\begin{enumerate}
\def\labelenumi{(\arabic{enumi})}
\setcounter{enumi}{3}
\tightlist
\item
  Calculating SI scores for a given time period: The si\_calc.R function
  calculates the number of arrivals in a given time period, using code
  adapted from an article included in the tidytransit package
  \citep{tidytransit_departure_timetable}, and combines this with the
  calculated area components. The si\_total.R and hourly.R functions
  provided aggregation, giving the results mapped in Figure
  @ref(fig:SI\_mornington\_20181230\_output).
\end{enumerate}

\begin{figure}
\centering
\includegraphics{Leveraging_GTFS_to_assess_transit_supply_Transport_Geography_files/figure-latex/SI_mornington_20181230_output-1.pdf}
\caption{Mornington Penninsula Tourist Railway hourly SI values for
December 30, 2018}
\end{figure}

\hypertarget{central-melbourne}{%
\subsection{Central Melbourne}\label{central-melbourne}}

Figure @ref(fig:Melbourne\_CBD\_map\_230808) SI scores for Tuesday
August 8, 2023 by mode for SA1 zones in Central Melbourne. SI scores are
generally higher in the Central Business District (CBD). Results are
consistent with: the high number for bus services along the Victoria and
Queen Street corridors; the rail stations surrounding the CBD; and tram
services that mostly run along the Swanston, Elizabeth, Bourke and
Collins Street corridors.

\begin{figure}
\centering
\includegraphics{Leveraging_GTFS_to_assess_transit_supply_Transport_Geography_files/figure-latex/Melbourne_CBD_map_230808-1.pdf}
\caption{Victorian GTFS and central Melbourne SA1 zones, SI values for
October 10, 2023, by mode}
\end{figure}

\hypertarget{greater-melbourne}{%
\subsection{Greater Melbourne}\label{greater-melbourne}}

\begin{tabular}{l|c}
\hline
**Characteristic** & **N = 40**\\
\hline
Population & \\
\hline
Median (IQR) & 104,332 (77,574, 156,285)\\
\hline
Skew & 1\\
\hline
Mean (SD) & 118,389 (59,360)\\
\hline
10\%, 90\% & 52,168, 193,124\\
\hline
5\%, 95\% & 41,080, 218,426\\
\hline
1\%, 99\% & 25,502, 264,363\\
\hline
Range & 25,146, 279,213\\
\hline
SI & \\
\hline
Median (IQR) & 94,230 (53,209, 136,011)\\
\hline
Skew & 5\\
\hline
Mean (SD) & 132,633 (208,439)\\
\hline
10\%, 90\% & 17,527, 220,764\\
\hline
5\%, 95\% & 12,483, 289,046\\
\hline
1\%, 99\% & 5,282, 938,342\\
\hline
Range & 839, 1,335,114\\
\hline
\end{tabular}

\begin{figure}
\centering
\includegraphics{Leveraging_GTFS_to_assess_transit_supply_Transport_Geography_files/figure-latex/greater_melbourne_230808_by_sa3_population_and_si_plot-1.pdf}
\caption{Victorian GTFS and SA3 zones within Greater Melbourne, SI
values for Tuesday August 8, 2023 and 2021 census population}
\end{figure}

\begin{figure}
\centering
\includegraphics{Leveraging_GTFS_to_assess_transit_supply_Transport_Geography_files/figure-latex/greater_melbourne_230808_by_gini_and_scatter-1.pdf}
\caption{Victorian GTFS and SA3 zones within Greater Melbourne, SI
values for Tuesday August 8, 2023 and 2021 census population}
\end{figure}

\hypertarget{by-year}{%
\subsection{By year}\label{by-year}}

\hypertarget{extensions}{%
\section{Extensions}\label{extensions}}

\hypertarget{melbourne-cbd-index}{%
\subsection{Melbourne CBD Index}\label{melbourne-cbd-index}}

\hypertarget{new-york-index}{%
\subsection{New York Index}\label{new-york-index}}

\hypertarget{london-index}{%
\subsection{London Index}\label{london-index}}

\hypertarget{discussion-and-conclusions}{%
\section{Discussion and conclusions}\label{discussion-and-conclusions}}

\renewcommand\refname{References}
\bibliography{References.bib, packages.bib}


\end{document}
